\documentclass{article}
\usepackage[utf8]{inputenc}
\usepackage{glossaries}
\usepackage{amsmath}
\usepackage[german]{babel}
\usepackage{hyphenat}

\def\changemargin#1#2{\list{}{\rightmargin#2\leftmargin#1}\item[]}
\let\endchangemargin=\endlist 

\begin{document}

\title{Besondere Lernleistung - DMergency}

\author{Mattis Rinke}

\date{März 2022}


\titlepage{
    \vspace*{\fill}
    \begin{center}
        \large{
            \begin{center}
                Gymnasium Nepomucenum Coesfeld \\
            \end{center}
            
            Besondere Lernleistung im Abitur 2022
        }
    \end{center}
    \;
    \begin{center}
        \LARGE{ \textbf{DMergency - App zur Alarmierung und Verwaltung von Sanitätsdiensten\\}}
    \end{center}
    \vspace{0,1cm}
    \begin{center}
        \large Entwicklung einer App zur effizienten und benutzerfreundlichen Alarmierung, sowie Verwaltung von Sanitätsdiensten
    \end{center}

    \begin{center}
        \vspace{2cm}
        \Large vorgelegt von\\
        \vspace{3cm}
        \LARGE{\textbf{Mattis Rinke}}
    \end{center}

    \begin{center}
        \Large{
            \vspace{3cm}
            Fachbereich Informatik\\
            \vspace{1cm}
            Herr Brumma\\
            Herr Willenbring
        }
    \end{center}
    \vspace*{\fill}


}
\newpage
\tableofcontents
\newpage

\section{Einleitung}

\begin{changemargin}{0,5cm}{0,0cm} 
    In der heutigen Zeit wird der Drang nach Digitalisierung immer stärker. So auch in Sanitätsdiensten, die darauf angewiesen sind schnell und effizient alarmiert zu werden.
    Diesen Drang habe ich selbst in meiner Funktion als Schulsanitätsdienst-Leiter und als Mitglied beim DRK mitbekommen. Da ich bereits das ein oder andere kleinere Projekt selbst 
    programmiert hatte, habe ich mir überlegt selber eine App zum Alarmieren und Verwalten von Sanitätsdiensten zu programmieren, was sich am Ende als deutlich schwieriger und komplexer 
    rausstellte als am Anfang erwartet.

\end{changemargin}

\section{Wahl der Entwicklungsweise}
    \subsection{Version 1 - Native Entwicklung}
    \subsection{Version 2 - Xamarin Forms}
    \subsection{Version 3 - Flutter}
        Erklärung warum Flutter genutzt wird

\section{Das Projekt}
\subsection{Zielsetzung}
\begin{changemargin}{0,5cm}{0,0cm}
    Ziel des Projekts ist das effiziente und sinnvolle Alarmieren, sowie Verwalten
    von Sanitätsdiensten. Hierbei soll zum einen die Nutzerfreundlichkeit, als auch
    die Effizienz und Kompatiblität im Vordergrund stehen. 
    \\Hierbei geht es grundlegend um die Funktionen der Alarmierung, der Dienstregelungen (Wann hat welcher Sanitäter Dienst) und den Empfang des Alarms.
    \\Dabei soll die Alarmierung, so intuitiv wie möglich und komplex wie nötig gestalltet werden um den Sanitätern so viele
    Informationen wie möglich zu geben, den Alarmierungsprozess für den Alarmierenden jedoch nicht unnötig kompliziert zu gestalten.
    Dies ist vor allem unter dem Punkt zu sehen, dass größtenteils von Laien alarmiert werden soll, da sich das Projekt auf 
    kleine Sanitätsdienste bzw. Schulsanitätsdienste beschränkt.
    \\ Um dies Umzusetzen soll eine App programmiert werden, die sich schnell öffnen lässt und von den Alarmierenden schnell
    genutzt werden kann um die nötigsten Informationen einzugeben und dann schnell zu Alarmieren.
    Des weiteren soll es in der gleichen App möglich sein die gesendeten Alarme zu empfangen und dann eine Rückmeldung an den betroffenen Alarmierenden zu senden.
    Damit der Sanitäter nicht dauerhaf im Dienst ist soll es außerdem einen Dienstplan geben, aus dem man sich im Notfall austragen kann oder aber auch den Dienst 
    eines anderen Sanitäters zu übernehmen, falls dieser verhindert ist.
\end{changemargin}

\subsection{Abgrenzung zur Server Ausarbeitung}
\begin{changemargin}{0,5cm}{0,0cm}
    In dieser Ausarbeitung geht es um die App "Dmergency". Nicht um die Webanwendung, bzw. den Server.
    Es wird jedoch auf die Kommunkation mit dem Server eingegangen um den Datenfluss darzustellen und die
    Funktionsweise der App zu verdeutlichen. 
\end{changemargin}

\subsection{Funktionen der App}
    \begin{changemargin}{0,5cm}{0,0cm}
        In den folgenden Abschnitten werden jetzt die Funktionen der App dargestellt und erklärt.
        Dazu werden beispielhaft einzelne Methodenimplementationen herrausgenommen, erörtert und
        im Kontext der jeweiligen Funktion erklärt.   
    \end{changemargin}
    \subsubsection{Rollen und Registrierung}
    \subsubsection{Alarmauslösung}
    \subsubsection{Alarmempfang}
    \subsubsection{Vertretungen}
    \subsubsection{News}
    \subsubsection{Notfallnummern}

\section{Kommunikation mit dem Server}
\subsection{API-Nutzung}
\subsection{Nutzung von Firebase-Messaging}
\section{Speicherung der Daten}
\subsection{Umsetzung}
\begin{changemargin}{0,5cm}{0,0cm}
    Es gibt mehrere Möglichkeiten auf mobilen Endgeräten appspezifische Daten zu speichern.
    Zum einen gibt es die so genannten SharedPreferences bzw. NSUserDefaults\footnote{SharedPreferences (Android), bzw. NSUserDefaults(iOS) ist platformspezifischer Langzeitspeicher für einfache Daten (String, Integer)}
    dies sind einfache Schlüssel, mit denen ein Wert verknüpft wird.
    Eine weitere Möglichkeit ist eine lokale Datei, in welche alle wichtigen Daten geschrieben werden, oder als letzte, dritte Möglichkeit gibt es
    die Datenbank.

    Im Fall der App DMergency habe ich zunächst versucht die Platform-Nativen Speichermethoden, also die SharedPreferences bzw. die NSUserDefaults, zu nutzen.
    Dies habe ich gemacht, da ich in der App selbst eigentlich kaum Daten speichern muss, was im Laufe der Ausarbeitung noch deutlich wird.
    Da die Platform-Nativen Speichermethoden so einfach gehalten sind und in der App nur Strings und Integer gespeichert werden müssen bieten sich diese also sehr gut an.

    Dieser Gedanke musste jedoch schnell verworfen werden, da die Platform-Nativen Speichermethoden unterschiedliche Instanzen in unterschiedlichen Threads haben.
    \\
    \\Dies wirft zwei Probleme auf:
    \item 1. Die Effizienz der App wird stark beeinträchtigt
    \item 2. Daten können in unterschiedlichen Threads nicht mit den Platform-Nativen Speichermethoden abgerufen werden.
    \\\\Um Alarme zu empfangen muss ein Hintergrundprozess laufen, welcher in einem anderen Thread arbeitet als der Rest der App.
    Da jedoch auf die Alarmdaten auch im Rest der App zugegriffen werden muss und auch im Hintergrundprozess auf die gespeicherten Nutzerdaten zurückgegriffen werden muss, sind die
    platformspezifischen Speichermethoden für die App nicht in Frage gekommen.

    Daher habe ich mich dazu entschieden eine Datenbank anzulegen, welche als lokale Datei auf dem Handy abgelegt ist.
    Dadurch ist es mögich jederzeit auf alle Daten zuzugreifen und die Probleme, der Effizienz und der Speicherung der Daten sind behoben.
    

\end{changemargin}
\subsection{Aufbau der Datenbank}
\section{Design Umsetzung}

\section{Schultests}

\section{Fazit}
\subsection{Was wurde erreicht}
\subsection{Wie geht es weiter}
\cite{VeniceSpringer}

\section{Abbildungsverzeichnis}
\section{Literaturverzeichnis}
    \bibliographystyle{plain}
    \bibliography{Lernleistung}
\section{Selbstständigkeitserklärung}
\section{Anhang}

\newpage

\end{document}