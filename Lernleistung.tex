\documentclass[12pt]{article}
\usepackage[utf8]{inputenc}
\usepackage{glossaries}
\usepackage{amsmath}
\usepackage[german]{babel}
\usepackage{hyphenat}
\usepackage{graphicx}
\usepackage{setspace}
\usepackage{float}

\def\changemargin#1#2{\list{}{\rightmargin#2\leftmargin#1}\item[]}
\let\endchangemargin=\endlist 

\graphicspath{\main/images}

\begin{document}
\onehalfspacing

\titlepage{
    \vspace{1cm}

    \begin{center}
        \large{
            \begin{center}
                Gymnasium Nepomucenum Coesfeld \\
            \end{center}

            Besondere Lernleistung im Abitur 2022
        }
    \end{center}
    \vspace{0,1cm}
    \begin{center}
        \includegraphics[width = 5cm, height = 2.5cm]{images/Nepo_Logo.png}
    \end{center}
    \vspace{0,1cm}
    \begin{center}
        \Large{ \textbf{DMergency - App zur Alarmierung und Verwaltung von Sanitätsdiensten\\}}
    \end{center}
    \vspace{0,1cm}
    \begin{center}
        \large Entwicklung einer App zur effizienten und benutzerfreundlichen Alarmierung, sowie Verwaltung von Sanitätsdiensten
    \end{center}

    \begin{center}
        \vspace{0,5cm}
        \large vorgelegt von\\
        \vspace{1cm}
        \LARGE{\textbf{Mattis Rinke}}
    \end{center}

    \begin{center}
        \large{
            \vspace{2cm}
            Fachbereich Informatik\\
            \vspace{1cm}
            Herr Brumma\\
            Herr Willenbring
        }
    \end{center}
    \vspace*{\fill}

}
\newpage
\thispagestyle{empty}
\tableofcontents
\thispagestyle{empty}

\newpage
\section{Einleitung}
\setcounter{page}{4}
    In der heutigen Welt wird der Drang nach Digitalisierung immer größer, wie auch bei Sanitätsdiensten.
    Die habe ich selbst durch meine Tätigkeit im Schulsanitätsdienst erfahren. Hier wurde bisher meist mit
    Funkgeräten oder auch mit Schuldurchsagen alarmiert, was den Schulunterricht drastisch gestört hat.
    Außerdem ist die Alarmierung selbst sehr ineffizient, da zunächst eine Person im Sekretariat oder ähnlichem 
    informiert werden muss, die anschließend dann die Sanitäter/-innen alarmiert.\\ 
    Um hier Verbesserung zu schaffen, habe ich mich dann nach Apps für eine einfache Alarmierung umgeguckt. Dabei habe ich dann
    mehrere Apps gefunden unter anderem die Apps "Sanialarm" und "Divera 24/7", beide haben die grundsätzlichen Voraussetzungen um 
    für den Schulsanitätsdienst genutzt zu werden: Beide Apps können Alarmieren und einen Dienstplan erstellen.
    Jedoch haben beide auch ihre Einschränkungen, wodurch diese nicht komplett für den Schulsanitätsdienst geeignet sind. 
    Zum Beispiel ist die Alarmierung von Sanialarm nicht zuverlässig und die Erstellung des Dienstplans muss bei Divera für jeden Tag einzeln erfolgen.
    Aufgrund dieser Einschränkungen habe ich mich zusammen mit einem Freund darangesetzt diese Problem zu lösen.
    Durch meine Vorkenntnisse im Fach Informatik bin ich dann schnell auf die Idee gekommen
    die Alarmierung per App zu gestalten. 
    Ich habe mich dann, als ich bereits angefangen hatte die App zu programmieren von der Möglichkeit erfahren
    eine besondere Lernleistung in das Abitur einfließen zu lassen, dazu entschieden dies zu tun.\\
    In dieser Ausarbeitung gehe ich darauf ein, wie die App entstanden ist, warum ich mich für das Framework Flutter
    entschieden habe, welche Funktionen die App hat und wie diese umgesetzt wurden.\\
    Als erstes werde ich skizzieren, was die App warum können soll und beschreibe im Anschluss, wie
    ich mich von der nativen Entwicklung zur Nutzung des Cross-Platform-Frameworks Flutter entschieden habe.
    Danach erkläre ich detaillierter die einzelnen Funktionen der App, woraufhin die Kommunikation mit dem 
    Server und die Speicherung der Daten näher erörtert wird. Zuletzt führe ich noch Ergebnisse der ersten 
    Schultests auf und ziehe dann ein Fazit, in dem ich erkläre, was in der Ausarbeitung bereits geschafft 
    wurde und wie ich mit der App weiterhin verfahre.

\section{Das Projekt}
\subsection{Zielsetzung}
    Die App soll das Alarmieren und Verwalten von Sanitätsdiensten vereinfachen. Um dies
    zu verwirklichen ist es erforderlich mehrere Funktionen zu implementieren. Zum 
    einen ist eine Funktion zum Alarmieren unerlässlich, welche zur Vergewisserung für
    die alarmierende Person auch ein Feedback anzeigen sollte, zum anderen sollte es 
    dann auch eine Funktion zum Empfangen des Alarms geben. Diese beiden Funktionen 
    sollten so implementiert werden, dass eine alarmierende Person so wenig Aufwand wie 
    möglich in der Durchführung der Alarmierung hat, die Sanitäter/-innen jedoch trotzdem 
    so viele Informationen wie möglich bekommen. Damit eine Verständigung seitens der 
    Sanitäter/-innen darüber erzielt werden kann, wer sich für das Einsatzmaterial 
    verantwortlich zeigt, ist es erforderlich, dass dies ebenfalls mit in die Funktionen 
    aufgenommen wird.
    Darüber hinaus ist die Programmierung eines Dienstplan-Systems vorgesehen, das nur
    diensthabende Sanitäter/-innen alarmiert.
    Selbstverständlich gibt es im (Schul-)Alltag auch Situationen, in denen Sanitäter/-innen
    kurzfristig (z.B. im Falle von Krankheiten) oder auch absehbar längerfristig (z.B.
    durch angekündigte Arbeiten, Ausflüge oder Praktika) ihren Dienst nicht durchführen 
    können, so dass es eine Funktion geben muss, mit der sich diese austragen können und
    andere ihren Dienst übernehmen.
    Im Notfall ist eine schnelle Rettungskette von großer Bedeutung. Um diese Hilfskette
    \cite{Rettungskette} einzuhalten, sollen in der App wichtige Notfallnummern hinterlegt
    werden, welche dann durch einen schnellen Klick wählbar sind.
    
    Ein weiterer wichtiger Bestandteil zur Verwaltung des Sanitätsdienstes 
    ist die Kommunikation zwischen der Leitung und den Mitgliedern des Sanitätsdienstes.
    Um diese Kommunikation sicherzustellen soll eine News-Funktion programmiert werden, 
    in welcher die Mitglieder Neuigkeiten von der Leitung einsehen können.
    Hierzu ist es unabdingbar eine Funktion umzusetzen, die es der Leitung ermöglicht,
    "News schreiben" zu können.

\subsection{Abgrenzung zur Server Ausarbeitung}
    In dieser Ausarbeitung wird die Funktionsweise der App \glqq DMergency" \grqq{} 
    beschrieben und wie sie in Zusammenarbeit mit dem Server arbeitet. Es wird nicht 
    darauf eingegangen, wie der Server arbeitet und welche Funktionen es in der 
    Web-Anwendung gibt. Zum Teil werden Daten vom Server verarbeitet oder auf diesem 
    gespeichert. In diesen Fällen wird dies erwähnt jedoch nicht weiter auf die
    Verarbeitung eingegangen.
\section{Wahl der Entwicklungsweise}
Es gibt in der Programmierung etliche Möglichkeiten der Programmierung. Auch in der Entwicklung für mobile Endgeräte.
Als ich mit dem Projekt angefangen habe, habe ich zunächst den mir am sinnvollsten erscheinenden Weg genommen. 
Die native Entwicklung. Dadurch, dass die App für unterschiedliche Betriebssysteme erhältlich sein soll, 
muss dies in dem Fall dann zweimal geschehen. Einmal für das Betriebssystem Android\cite{Android}, von Google, und für
das Betriebssystem iOS\cite{iOS} von Apple. Es gibt aber auch die Möglichkeit der Cross-Platform-Programmierung, bei 
der für beide Betriebssysteme gleichzeitig programmiert wird. Ich habe mich zunächst für die native Programmierung 
entschieden, jedoch habe ich mich im Entwicklungsprozess von der nativen Entwicklung zur Cross-Platform-Programmierung 
mit dem Framework Xamarin-Forms bewegt um schließlich das Framework Flutter zu verwenden. Was die Vor- und Nachteile
sind und warum ich mich letztendlich für die Cross-Platform-Programmierung mit Flutter entschieden habe erkläre ich in 
den nächsten drei Abschnitten.

    \subsection{Version 1 - Native Entwicklung}
    Zunächst habe ich mit der nativen Entwicklung von Android begonnen, da ich die eine garantierte Kompatibilität
    mit dem gewünschten Betriebssystem habe. Dies stellte sich als einfach heraus, da ich hier in der 
    Programmiersprache Java schreiben muss, die ich bereits aus dem Informatik-Unterricht kannte.
    Hier hatte ich dann nach einiger Zeit eine vor**läufig fertige App programmiert, in welcher
    man einen Alarm auslösen und empfangen konnte.
    Da der Markt zwischen Apple und Google in Sachen Handy-Betriebssysteme bei ca. 70\% zu 30\% 
    liegt\cite{Marktanteil}, habe ich schnell gemerkt, dass ich die App auch für iOS entwickeln muss.
    Um alle Funktionen immer auf jeder Platform verfügbar zumachen muss bei dieser Entwicklungsmethode
    jede Funktion zweimal programmiert werden. Dies ist für mich als Einzelperson nicht machbar, weshalb
    ich mich neu orientieren musste. Ich habe mich weiter informiert und habe dann die Methode des 
    Cross-Platform-Programmings gefunden, auf welche ich in den nächsten zwei Abschnitten eingehen werde.
    \subsection{Version 2 - Xamarin Forms}
    Xamarin Forms ist ein Framework der .NET-Platform von Microsoft. Dieses Framework ist ein 
    Cross-Platform-Framework, das heißt, dass der Code einmalig für die beiden Betriebssysteme (Android \& iOS)
    geschrieben wird und die App dann für beide erhältlich ist.
    Xamarin Forms hat eine Unterteilung zwischen dem funktionalen Code, welcher in C\# geschrieben ist, und zwischen
    der der Markup Language XAML, welche das GUI darstellt\cite{Xamarin}.
    Jedoch sind bei der Nutzung von Cross-Platform-Frameworks auch Einbußungen zu machen. In diesem Fall konnte 
    ich mich zum einen nicht mit der Markup Language XAML anfreunden, zum anderen musste ich jedoch auch 
    herausfinden, dass Xamarin einige von mir benötigte Funktionen nicht voll oder gar nicht unterstützt. 
    Unter anderem gab es immer wieder Probleme beim einbinden von Firebase-Messaging, ein Tool von Google,
    zum versenden von Push-Notifications (Auf Firebase-Messaging gehe ich im Laufe der Ausarbeitung noch ein).
    Durch diese für mich nicht lösbaren Probleme musste ich mich dann erneut auf die Suche nach einer anderen
    Lösung machen. Um diese Lösung geht es jetzt im nächsten Abschnitt.
    \subsection{Version 3 - Flutter}
    Die dritte und bis jetzt finale Version ist in Flutter geschrieben und die am weitesten entwickelte App-Version.
    Die Entscheidung für Flutter viel nach mehreren Empfehlungen, sowie nach eigener Recherche. 
    Flutter nutzt eine einfache Programmiersprache, die Java ähnelt, weshalb mir der Einstieg in die
    Programmiersprache Dart\cite{Dart} nicht schwer gefallen ist. Flutter ist wie Xamarin-Forms ein 
    Cross-Platform Framework. Dieses ist in der Lage Apps für die beiden gängigen Plattformen iOS und Android 
    zu kompilieren\footnote{Kompilieren beschreibt das Umwandeln des geschriebenen Programmtextes in ein 
    funktionsfähiges Programm}, sowie für das Web.
    Das Nutzer-Interface wird in Flutter durch so genannte Widgets, welche in einem Widget-Tree 
    aufgenommen werden, dargestellt. Hierbei wird dann zwischen Stateless\cite{Stateless-Widget}-
    und Stateful\cite{Stateful-Widget}-Widgets unterschieden. Der Unterschied hier ist, dass in Stateless-Widgets
    zwar Variablen, etc. abgeändert werden können, jedoch wird das Widget im UI nicht aktualisiert. 
    Es können aber in dem Widget selbst andere Widgets, die selbst Stateful-Widgets sind, aktualisiert werden.
    Ein Stateful-Widget besteht aus zwei Klassen. Der Klasse, die um die Klasse Stateful-Widget erweitert wird 
    und der Klasse, die um die Klasse State erweitert wird. In der Klasse mit dem Stateful Widget wird in der
    createState-Methode die Klasse mit dem State zurückgegeben. Der eigentliche Widget-Tree wird dann erst in
    der build-Methode der State-Klasse geschrieben und dann auch zurückgegeben.
    Nach dem der Widget-Tree dann erstellt wurde kann bei Bedarf durch die Methode setState das UI aktualisiert werden.
    Außerdem gibt es asynchrone Methoden und Futures\cite{Futures}.
    Diese sind dazu da das Programm weiterhin Code ausführen zu lassen, während dieses gleichzeitig darauf 
    wartet, dass die asynchrone Methode fertig wird. Dies wird oft dazu genutzt Daten vom Netzwerk zu laden
    oder in eine Datenbank zu schreiben. Die meist genutzten Datentypen sind: int, String, bool, List, Future und Map.
    Ich habe mich nach einiger Überlegung und Recherche dann letztendlich für Flutter als Framework 
    entschieden, weil ich zum einen nur einen Code für die beiden Plattformen iOS und Android schreiben muss 
    und zum anderen da das UI einfach programmierbar ist und Flutter sehr performant ist.

\section{Entwicklung}
\subsection{Funktionen der App}
    \begin{changemargin}{0,5cm}{0,0cm}
        In den folgenden Abschnitten werden jetzt die Funktionen der App dargestellt und erklärt.
        Dazu werden beispielhaft einzelne Methodenimplementationen herrausgenommen, erörtert und
        im Kontext der jeweiligen Funktion erklärt.   
    \end{changemargin}
    \subsubsection{Rollen und Registrierung}
        \begin{changemargin}{0,5cm}{0,0cm}
            Rolle 1: Alarmierende\;

            Die Rolle Alarmierende/r soll nur dazu in der Lage sein einen Alarm auszulösen, 
            sowie die News für Alarmierende und Notfallnummern einzusehen.\;
            Ein Alarmierender muss bei der Registrierung einen Namen, eine E-Mail-Adresse, sowie ein
            Passwort angeben.
            Die E-Mail-Adresse wird verwendet um den Account innerhalb der App zu identifizieren und 
            den Nutzer in der App einzuloggen. Das Passwort um sich jederzeit an einem Handy
            einloggen zu können. Der Name dient der / den administrierenden Person(en) zur
            Identifikation des Alarmierenden.
        \end{changemargin}

        \begin{changemargin}{0,5cm}{0,0cm}
            Rolle 2: Sanitäter/-in\;

            Die Rolle Sanitäter/-in soll in der Lage sein einen Alarm zu empfangen, einen Alarm
            auszulösen und andere Sanitäter/-innen zu vertreten oder sich im Notfall aus dem Dienst 
            auszutragen.
            Ein/e Sanitäter/-in muss bei der Registrierung seinen/ihren Vornamen,  Nachnamen,
            das Geschlecht, eine E-Mail-Adresse, ein Passwort, sowie eine Stufe angeben.

        \end{changemargin}

        \begin{changemargin}{0,5cm}{0,0cm}
            Um die unterschiedlichen Rollen umzusetzen muss auch der Registrierungsprozess von
            den verschiedenen Rollen verscieden ablaufen.
        \end{changemargin}

    \subsubsection{Alarmauslösung}
    \subsubsection{Alarmempfang}
    \subsubsection{Vertretungen}
    \subsubsection{News}
    \subsubsection{Notfallnummern}
\subsection{Kommunikation mit dem Server}
Um alle Funktionen anbieten zu können müssen einige Funktionen auf einen Server ausgelagert werden.
Unter anderem werden hier die Sanitätsdienste mit ihren Sanitäter/-innen und Alarmierenden verwaltet.
Im Folgenden werde ich nun zu erst die API-Nutzung darlegen und im Anschluss Firebase-Messaging erklären, sowie auf die 
Implementierung von Firebase-Messaging eingehen.
\subsubsection{API-Nutzung}
Die API funktioniert so, dass ich zunächst eine Request, also eine https-Anfrage, an den Server sende.
Hierbei spezifiziere ich zunächst den Pfad (/path) und danach werden Attribute in der query \\(?attribut1\=wert1\&attribut2\=wert2) angegeben. Diese bestehen meistens
aus der Sanitätsdienst-ID, der Nutzer-ID, der Nutzer-Rolle und weiteren spezifischen Attributen je nach Daten, die abgefragt werden sollen.

\noindent Der Aufbau einer solchen Anfrage würde dann wie folgt aussehen: 
%Bild einfügen: https://www.dmergency.de/Alarmfeedback?schoolid=2&ssdid=2&typeid=0&alid=4
%               schema      domain          path        attributes      
Nach dem Empfang der Daten vom Server wird die Antwort des Servers zunächst in einem JSON-Array gespeichert um dann weiter verarbeitet zu werden.
Dieser ist dann je nach Art der benötigten Daten 1-3 Dimensional. Zum Beispiel ist der JSON-Array für die Berechtigungen 1 Dimensional, der JSON-Array für die 
Sanitäter/-innen, die aktuell Dienst haben 2 Dimensional usw.



\subsubsection{Nutzung von Firebase-Messaging}
Firebase-Messaging wird genutzt um die Alarme die, ausgelöst und vom Server verarbeitet werden, vom Server an die Smartphones der Sanitäter/-innen zu schicken.
Hierzu habe ich mich entschieden, da Firebase-Messaging mit den beiden Plattformen, iOS und Android, für welche auch meine App erhältlich ist, kompatibel ist und ich daher
ohne Probleme Push-Notifications an die Smartphones der Sanitäter/-innen geschickt werden können. Da der Server, dies auch unterstützt und dafür die Möglichkeit bereitstellt, war dies 
schnell umgesetzt.

\noindent Firebase-Messaging ist ein Cloud-Messaging Service von Google. Dieser Service ist in der Lage Push-Notifications an 
Clients zu schicken. Dieses erfolgt über einen sogenannten FCM(Firebase-Cloud-Messaging)-Token, welcher einem spezifischen Gerät
bei der Installation der App zugewiesen wird. Jeder FCM-Token ist einzigartig und wird von Zeit zu Zeit auf jedem Gerät aktualisiert.\cite{FCM-Update}
Das aktualisieren des Tokens kann durch mehrere Ereignisse ausgelöst werden. Zum einen dies dadurch ausgelöst werden, wenn die App auf einem anderen 
Gerät wieder hergestellt wird, oder der Nutzer die App deinstalliert bzw. diese reinstalliert, oder wenn der Nutzer die App-Daten löscht.

\noindent Firebase-Nachrichten sind grundlegend immer gleich aufgebaut. Grundlegend sind Firebase Nachrichten JSON-Arrays. Diese haben immer einen
message-Teil, dieser kann dann noch weiter aufgedröselt werden. Ein wichtiger Teil, der in fast allen Nachrichten enthalten ist, ist der notification-Teil. 
In diesem wird dann der Notification-title und Notification-body angegeben, welche in der Notification angezeigt wird.
Außerdem gibt es den data-Teil, welcher für die Daten Übermittlung zwischen Gerät und Server wichtig ist, da dieser selbst gestaltet werden kann.
Durch den Server ist hier vorgegeben, dass bei jedem Alarm eine AlarmId, ein Alarm-Ort, eine Alarm-Beschreibung, eine Alarm-Priorität, die Zeit der Alarmauslösung, sowie das Datum von dem Alarm.


\subsection{Speicherung der Daten}
\subsubsection{Umsetzung}
    Es gibt mehrere Möglichkeiten auf mobilen Endgeräten App-spezifische Daten zu speichern.
    Zum einen gibt es die so genannten SharedPreferences bzw. NSUserDefaults
    \footnote{SharedPreferences (Android), bzw. NSUserDefaults(iOS) ist plattformspezifischer Langzeitspeicher für einfache Daten (String, Integer)}
    dies sind einfache Schlüssel, mit denen ein Wert verknüpft wird.
    Eine weitere Möglichkeit ist eine lokale Datei, in welche alle wichtigen Daten geschrieben werden, oder als letzte, dritte Möglichkeit gibt es
    die Datenbank.

    Im Fall der App DMergency habe ich zunächst versucht die Platform-Nativen Speichermethoden, also die SharedPreferences bzw. die NSUserDefaults, zu nutzen.
    Dies habe ich gemacht, da ich in der App selbst eigentlich kaum Daten speichern muss, was durch die später folgende Skizzierung des Aufbaus der letztendlich genutzten Datenbank deutlich wird.
    Die SharedPreferences und NSUserDefaults sind generell sehr einfach gehalten, da die Daten über einen eindeutigen String identifiziert werden. Das heißt also es gibt einen "Schlüssel", der zu einem Wert zugeordnet wird.
    Die SharedPreferences bzw. NSUserDefaults können als Datentypen dann entweder einen String, einen int oder einen boolean zugewiesen bekommen.
    Wie bereits erwähnt habe ich zunächst diese Methode verwendet, da sie zum einen einfach ist, zum anderen aber auch kaum Daten gespeichert werden müssen.

    Diese Methode musste ich jedoch schnell wieder verwerfen, da Flutter mit sogenannten Isolates arbeitet. Diese sind etwas ähnliches wie Threads, was bewirken soll, dass mehrere Funktionen gleichzeitig laufen können.
    Jedoch haben die SharedPreferences bzw. die NSUserDefaults dann pro Isolate eine eigene Instanz, was bedeutet, dass die Daten, die in der einen Isolate gespeichert wurden nicht von der anderen Isolate aus bearbeitet werden kann oder ähnliches.
    Dies ist ein Problem, da die Alarme, durch den Cloud-Messaging-Dienst Firebase-Messaging empfangen werden, dieser arbeitet dauerhaft in einer anderen Isolate als die Hauptisolate der App, da diese dauerhaft im Hintergrund laufen muss um die Alarme
    empfangen zu können. 
    Da ich aber auf die Daten, die durch Firebase-Messaging gesendet werden angewiesen bin um diese in der App anzeigen zu können muss eine andere Lösung gefunden werden, um die Daten zu speichern, da ich keinen direkten Zugriff auf die Instanz von Firebase-Messaging habe.

    \noindent Dadurch habe ich überlegt, dass ich die Daten dann in eine lokale Datei schreibe, um dann auf diese von jeder Instanz aus zugreifen zu können. Da dies jedoch eher Umständlich ist habe ich dann nach weiteren Möglichkeiten geguckt und bin zu dem Entschluss gekommen, dass die beste Möglichkeit, die Daten 
    zu speichern, ist, diese in einer lokalen Datenbank-Datei zu speichern und die Daten dann von den jeweiligen Isolates abzuändern oder aufzurufen.

    \noindent Dies ist letztendlich auch die Methode, die ich am Ende implementiert habe.
\subsubsection{Aufbau der Datenbank}

    Im folgenden Abschnitt werde ich jetzt skizzieren, welche Daten in speichere, warum ich diese speichere und wie ich darauffolgend die Datenbank aufgebaut habe.
    Um die Funktion der App gewährleisten zu können müssen Daten des Nutzers gespeichert werden. 
    Hierbei wird natürlich auf die Datenschutzbestimmungen geachtet.
    Diese besagen, dass alle erhobenen Daten nur für ihren angegebenen Zweck genutzt werden dürfen (Zweckbindung) und nur so viele Daten erhoben werden sollen wie benötigt werden(Datenminimierung)\cite{DSGVO}.
    In meinem Fall speichere ich zunächst einmal den Namen, bzw. den Nutzernamen des Nutzers / der Nutzerin, um der Sanitätsdienst-Administration zu ermöglichen die Nutzer/-innen zu identifizieren.
    Zudem wird die E-Mail des/der Nutzer/-in gespeichert, um die Nutzer/-innen eindeutig im System identifizieren zu können, daher: Die E-Mail-Adresse ist im gesamten Sanitätsdienst einzigartig.
    Außerdem wird das Geschlecht und die Qualifikation der Sanitäter/-innen gespeichert um den Administrator/-innen eine gute Dienstplanung zu ermöglichen.
    Außerdem wird zu jedem/-r Nutzer/-in die zugehörige Rolle gespeichert, also ob sie Sanitäter/-in oder Alarmierende/-r sind.
    
    \noindent Es werden jedoch nicht nur die Login-Daten gespeichert, sondern auch die Daten der Alarme. Auf den Smartphones der Sanitäter/-innen werden jedoch nur die Daten gespeichert, die für den/die Sanitäter/-in
    relevant sind. Dies ist zum einen die Alarm-ID, welche zur eindeutigen Identifikation des Alarms benötigt wird, zum anderen werden aber auch der Alarmierungszeitpunkt(Datum \& Zeit), die Beschreibung, der Ort und die Priorität gespeichert, um
    dem/der Sanitäter/-in möglichst viele Informationen über die Alarmierung zu geben. Der Alarmierungszeitpunkt ist zum Beispiel wichtig, wenn der Alarm verzögert kommen sollte, sodass der/die Sanitäter/-in weiß, dass eventuell Eile geboten ist, da schon mehr Zeit vergangen ist als eigentlich sollte.
    Als letztes wird für jeden Alarm die Rückmeldung des/der jeweiligen Sanitäter/-in die eigene Rückmeldung gespeichert, also ob der Alarm empfangen, Bestätigt oder Abgelehnt wurde.

    %Ausführen, warum welcher Datentyp wofür verwendet wird.

    \vspace{5mm}
    \noindent Die Datenbank sieht dann wie folgt aus: 

    \vspace{5mm}
    \noindent Die Tabelle Alarme hat wie in der Abbildung zu sehen eine Alarm-ID, welche ein Integer und zugleich der Primärschlüssel ist, da die Alarme mit ihrer Alarm-ID eindeutig identifiziert wird und somit einzigartig pro Sanitätsdienst ist.
    Die Beschreibung, der Ort, die Priorität, die Zeit und das Datum des Alarms werden als TEXT(String) abgespeichert, da diese theoretisch jede beliebige Zeichenkette enthalten können (sollen).

    \noindent Die Tabelle Login hat eine Nutzer-ID, welche ein Integer und der Primärschlüs-sel ist, da die Nutzer-ID den Login eindeutig kennzeichnet und über diese der/die Nutzer/-in eindeutig identifiziert wird.
    Außerdem gibt es das Attribut role, welches die Rolle des Nutzenden beschreibt. Hier nutze ich den Datentypen Integer um zwischen Sanitäter/-innen (role = 0) und Alarmierenden (role = 1) zu unterscheiden.
    Ich habe keinen boolean gewählt, da ich die Möglichkeit offenlassen möchte auch die Unterscheidung zum Administrator zu ermöglichen, da wie ich später noch ausführen werde noch einige Funktionen implementiert werden sollen, die 
    bisher nicht implementiert sind.
    Des weiteren hat die Tabelle das Attribut loggedin, vom Datentyp TEXT, mit welchem ich überprüfe ob der Nutzer angemeldet ist oder nicht. Hierzu sollte eigentlich der Datentyp boolean verwendet werden, jedoch ist dieser von dem sqlite-package nicht supported\cite{sqlite-Datatypes}.
    Zudem gibt es eine sanID als Integer-Attribut, welches den Sanitätsdienst beschreibt, dem der/die Sanitäter/-in angehört.
    Des weiteren werden der Vorname, der Nachname die Qualifikation und das Geschlecht der Nutzerin / des Nutzers als TEXT gespeichert.
    Das letzte Attribut, welches gespeichert ist, ist volume, welches vom Datentyp REAL ist. In diesem speichere ich einen double, welcher die in den Einstellungen festgelegte Lautstärke für Alarme speichert.

    \noindent Dies kann in der unten aufgeführten Abbildung nochmal entnommen werden.

    \noindent Wie bereits durch meine Ausführungen deutlich geworden sein sollte, wurde bei jeder erhobenen Information darauf geachtet, dass nur die Daten erhoben werden, die 
    für die Funktion der App von Nöten sind, wodurch die DSGVO in der App eingehalten wird.
    Diese gibt vor, dass alle erhobenen Daten nur für den angegebenen Zweck genutzt werden dürfen und nur Daten erhoben werden sollten, die unbedingt genutzt werden müssen\cite{DSGVO}.
    %Datenschutz mit einbeziehen --> Nur die Daten die benötigt sind, etc.
\section{Schultests}
Um die Funktionalität der App sicherzustellen wird die App zur Zeit an 
mehreren Schulen getestet. Dazu wird für die Schulen ein Sanitätsdienst erstellt,
über welchen diese dann alle Funktionen testen können. Sollte dann ein Fehler
auftreten können die Nutzer/-innen diesen über ein Fehler-Formular melden. 
Jedoch kann man nicht nur Fehler melden sondern auch generell Feedback, über ein
dediziertes Fehler-Formular, abgeben. Dies dient der allgemeinen Verbesserung
der App in jeglichen Belangen, seien es fehlende Funktionen, Design-Abänderungen
oder nur Lob. Derzeit testen bereits 5 Schulen die App, wobei generell positives
Feedback zurückkam. Jedoch trat natürlich auch schon der ein oder andere Fehler 
auf, der dann jedoch durch schnelle Kommunikation mit den jeweiligen 
Ansprechpartner/-innen der Schulen behoben werden konnte.
Zum Beispiel wurde bereits die Wahrscheinlichkeit erhöht, in der ein Alarm 
ankommt. Hier gab es vor allem am Anfang größere Probleme, da immer wieder 
Sonderfälle aufgetreten sind, durch die Fehler aufgetreten sind. Dadurch, dass
in der App dann Fehler aufgetreten sind konnte der Alarm nicht abgespielt und 
angezeigt werden. Dies konnte dann durch die Rückmeldungen der Schulen behoben
werden. Andere Fehler waren beispielsweise auch Grafikfehler, bei denen dann
gewisse Grafikobjekte nicht korrekt angezeigt wurden oder abgeschnitten waren.
Dies konnte dann auch schnell durch die Rückmeldungen behoben werden.
Des weiteren wurde dann auch ein Fehler beim Anmelden gemeldet, durch welchen
man sich weiterhin registrieren konnte, jedoch eine Anmeldung mit einem 
existenten Account nicht möglich war.
Durch mehrere schnelle Tests war dann schnell klar, dass der Fehler durch die 
falsche Verarbeitung der Serverantwort aufgetreten ist, sodass auch dieser 
schnell behoben werden konnte. 
Diese schnelle Fehlerbehebung war zum einen durch das bereits erwähnte 
Fehlerformular möglich, jedoch auch durch einen weiteren Service von Firebase.
Durch Firebase-Crashlytics ist es möglich, dass Fehler, die in der App 
auftreten auf einer Web-Oberfläche angezeigt werden. Durch die Implementation 
von Firebase-Crashlytics ist durch die Fehler-Meldungen eine Lösungsfindung 
deutlich einfacher, da hier der Fehler zum einen angezeigt wird und zum anderen
angezeigt wird, an welcher Stelle dieser ist.
\section{Fazit}
In den folgenden zwei Abschnitten werde ich jetzt ein Fazit ziehen, in dem ich zunächst aufzeige, was erreicht wurde und 
im Anschluss einen Ausblick gebe, was noch ansteht, beziehungsweise, welche Funktionen vielleicht noch hinzugefügt werden sollen o.ä.
\subsection{Was wurde erreicht}

\subsection{Wie geht es weiter}

\section{Abbildungsverzeichnis}
\listoffigures

\section{Literaturverzeichnis}

\begin{thebibliography}{9}
    \bibitem{Rettungskette}https://www.drk.de/hilfe-in-deutschland/erste-hilfe/rettungskette\\/rettungskette-uebersicht/
    \bibitem{Android}https://www.android.com/, Offizielle Android-Seite
    \bibitem{iOS}https://www.apple.com/de/ios/ios-15/, Offizielle iOS-Seite
    \bibitem{Marktanteil}https://de.statista.com/statistik/daten/studie/256790/umfrage\\/marktanteile-von-android-und-ios-am-smartphone-absatz-in\\-deutschland/\#:~:text=Marktanteile\%20von\%20Android\%20und\\\%20iOS,2021\&text=Im\%203\%2DMonatszeitraum\%20Oktober\\\%20bis,iPhone\%20betrug\%2030\%2C9\%20Prozent.
    \bibitem{Xamarin}https://docs.microsoft.com/en-us/xamarin/xamarin-forms/,Xamarin \\Forms Dokumentation
    \bibitem{Dart}https://dart.dev/, Dart Dokumentation
    \bibitem{flutter.dev}https://www.flutter.dev, Dokumentation des Flutter-Frameworks
    \bibitem{Stateless-Widget}https://api.flutter.dev/flutter/widgets/StatelessWidget-class.html,\\ Dokumentation der Klasse Stateless-Widget
    \bibitem{Stateful-Widget}https://api.flutter.dev/flutter/widgets/StatefulWidget-class.html, Dokumentation der Klasse Stateful-Widget
    \bibitem{Futures}https://dart.dev/codelabs/async-await, Dokumentation von Asynchronität
    \bibitem{FCM-Update}https://firebase.google.com/docs/cloud-messaging/manage-tokens, Aktualisierung von FCM-Token, Firebase-Dokumentation
    \bibitem{sqlite-Datatypes}https://pub.dev/packages/sqflite\#:~:text=Supported\%20SQLite\%20types\%20\%23, sqlite - supported Datentypen
    \bibitem{DSGVO}https://www.datenschutz-grundverordnung.eu/grundverordnung/art-5-ds-gvo/, DSGVO-Artikel 5
\end{thebibliography}

\newpage
\section{Anhang}

Hiermit erkläre ich, Mattis Rinke, dass ich diese Ausarbeitung ohne fremde Hilfe
angefertigt habe und nur die im Literaturverzeichnis aufgeführten
Quellen und Hilfsmittel genutzt wurden.\\

    ........................, den .................................................................................

\quad\quad  (Ort)    \hfil           (Datum)                   \hfil  (Unterschrift)
\end{document}