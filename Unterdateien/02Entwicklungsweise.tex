\section{Wahl der Entwicklungsweise}
   Es gibt etliche Möglichkeiten zu programmieren. Auch bei der 
    Entwicklung für mobile Endgeräte. Mit dem Projektstart wählte ich den mir zunächst am 
    sinnvollsten erscheinenden Weg der nativen Entwicklung.
    Dadurch, dass die App für unterschiedliche Betriebssysteme 
    erhältlich sein soll, muss die native Entwicklung sowohl für das Betriebssystem iOS 
    \cite{iOS} von Apple durchgeführt werden, als auch für das Betriebssystem Android 
    \cite{Android}, von Google.
    Es gibt aber auch die Möglichkeit der Cross-Platform-Programmierung, bei 
    der für beide Betriebssysteme gleichzeitig programmiert wird. 
    Wie schon erwähnt, entschied ich mich zunächst für die native Programmierung, jedoch 
    bewegte ich mich im Entwicklungsprozess von der nativen Entwicklung zur Cross-Platform-
    Programmierung mit dem Framework Xamarin-Forms, um schließlich das Framework Flutter zu 
    verwenden. Worin die Vor- und Nachteile liegen und warum ich mich letztendlich für die 
    Cross-Platform-Programmierung mit Flutter entschied wird in den nächsten drei 
    Unterkapiteln erläutert.

\subsection{Version 1 - Native Entwicklung}
    Der Einstieg in die native Entwicklung erfolgte mit Android, da hier für mich eine 
    garantierte Kompatibilität mit dem Betriebssystem vorlag. Diese stellte sich als mühelos
    heraus, da in der Programmiersprache Java geschrieben wird, die mir aus dem Informatik-
    Unterricht bereits bekannt war.
    Nach kurzer Zeit lag eine vorläufig fertig programmierte App vor, die einen zuverlässigen 
    Empfang und das Auslösen eines Alarms ermöglichte. Da diese App jedoch auch zuerst nur 
    für den Schulsanitätsdienst meiner eigenen Schule gedacht war, gab es hier kein Registrier-
    und Login-Verfahren, genauso wenig wie die Notfallnummern, die News und die Einstellungen.
    Diese App war also nur auf den Sanitätsdienst meiner eigenen Schule zugeschnitten.

    Da es jedoch auch iOS-Geräte in diesem Sanitätsdienst gibt, merkte ich schnell, 
    dass ich auch hier eine App programmieren muss. Also habe ich mich informiert,
    wie ich für iOS entwickeln und die App auch für iOS-Geräte veröffentlichen kann.
    Die Programmierung für iOS findet nativ in den Programmiersprachen Swift und Objective-C
    statt, was mich vor ein weiteres Problem stellte. Ich bin weder mit der Programmiersprache
    Swift noch mit der Programmiersprache Objective-C vertraut, was bedeuten würde, dass ich mir
    diese von null auf hätte beibringen müssen. Dies wäre zwar sehr unangenehm, jedoch möglich gewesen.
    Ich habe jedoch erst einmal weiter recherchiert und bin auf die nächste Barriere gestoßen:
    Um die App veröffentlichen zu können muss ich einen Apple-Developer-Account haben um die App
    im Appstore veröffentlichen zu können. Da dieser 99€ im Jahr kostet bin ich zu dem Entschluss 
    gekommen, dass es für mich nicht möglich ist die App nur für den Sanitätsdienst meiner eigenen Schule
    zu entwickeln. Damit ich dieses Problem umgehen konnte, habe ich mich dazu entschieden die App nicht 
    nur für meinen eigenen Sanitätsdienst zu entwickeln, sondern diese auch an weitere Sanitätsdienste
    zu verkaufen. Um dies zu bewerkstelligen musste ich jetzt also zum einen die App sowohl für
    Android, als auch für iOS zu entwickeln, da der Marktanteil von Apple bei 30\% und der von Google bei
    70\% liegen \cite[vgl.]{Marktanteil}. Bei der nativen Entwicklung müsste ich jetzt also 
    den Code für das \glqq gleiche\grqq{} Produkt zweimal schreiben, da ich die App einmal in
    Objective-C/Swift und einmal in Java/Kotlin(Kotlin ist eine weitere Programmiersprache, in der 
    es möglich ist Apps für Android zu programmieren) schreiben müsste.
    Dies ist als Einzelperson nicht machbar, weshalb ich mich neu orientierte. 
    Recherchearbeiten führten mich dann zu der Methode des Cross-Platform-
    Programmings, welche in den nächsten zwei Unterkapiteln erläutert werden.

\subsection{Version 2 - Xamarin Forms}

    Xamarin Forms ist ein Framework der sogenannten \glqq .NET-Plattform\grqq{} von Microsoft. Dieses 
    Framework ist ein Cross-Platform-Framework, das heißt, dass der Code einmalig für 
    die beiden Betriebssysteme (Android \& iOS) geschrieben wird und die App dann für 
    beide erhältlich ist. Xamarin Forms hat eine Unterteilung zwischen dem funktionalen 
    Code, welcher in der Programmiersprache C\# geschrieben ist, und zwischen der der Markup Language XAML, 
    welche die Benutzeroberfläche, oder auch das Graphical User Interface (GUI) genannt, darstellt\cite[vgl.]{Xamarin}. 
    Jedoch sind bei der Nutzung von Cross-Platform-Frameworks auch Einbußen zu erleiden. In diesem Fall konnte ich mich zum 
    einen nicht mit der Markup Language XAML anfreunden, da hier das Neuladen des GUIs 
    als sehr sperrig und aufwendig herausstellte, zum anderen stellte sich heraus, dass 
    Xamarin einige von mir benötigte Funktionen nicht voll oder gar nicht unterstützt. 
    Unter anderem gab es immer wieder Probleme beim Einbinden von Firebase-Messaging, einem
    Tool von Google zum Versenden von Push-Notifications (Auf Firebase-Messaging wird im
    weiteren Verlauf der Ausarbeitung noch eingegangen).
    Durch diese für mich nicht lösbaren Probleme musste ich mich dann erneut auf die Suche
    nach einer anderen Lösung machen. Um diese Lösung geht es jetzt im nächsten Abschnitt.

\subsection{Version 3 - Flutter}

    Die dritte und bis jetzt finale Version ist in Flutter geschrieben und die am weitesten 
    entwickelte App-Version. Die Entscheidung für Flutter fiel nach mehreren Empfehlungen, 
    z.B. durch einen Freund, welcher bereits durch ein Praktikum bei der Firma d.velop mit 
    diesem Erfahrung sammeln konnte, sowie nach eigener Recherche. Flutter nutzt eine 
    einfache Programmiersprache, die Java ähnelt, weshalb mir der Einstieg in diese 
    Programmiersprache, Dart\cite{Dart}, nicht schwergefallen ist. Flutter ist wie Xamarin-Forms
    ein Cross-Platform Framework. Dieses ist in der Lage, Apps für die beiden gängigen 
    Plattformen iOS und Android, sowie für das Web zu kompilieren (Kompilieren beschreibt das Umwandeln
    des geschriebenen Programmtextes in ein funktionsfähiges Programm).
    Das Nutzer-Interface ist in Flutter durch sogenannte Widgets, welche in einem \glqq Widget-Tree\grqq{} 
    aufgenommen werden, dargestellt. Hierbei wird dann zwischen \glqq Stateless\grqq
    \cite{Stateless-Widget}- und \glqq Stateful\grqq\cite{Stateful-Widget}-Widgets unterschieden. 
    Der Unterschied liegt darin, dass in \glqq Stateless-Widgets\grqq{} zwar Variablen, etc. abgeändert 
    werden können, jedoch wird das Widget im GUI nicht neu gerendert. Es können aber in dem 
    Widget selbst andere Widgets, die selbst \glqq Stateful-Widgets\grqq{} sind, aktualisiert werden.
    Ein Stateful-Widget besteht aus zwei Klassen. Der Klasse, die um die Klasse 
    \glqq Stateful-Widget\grqq{} erweitert wird und der Klasse, die um die Klasse 
    \glqq State\grqq{}\cite{State} erweitert wird. In der Klasse mit dem Stateful Widget wird in der
    \glqq createState\grqq{}-Methode die Klasse mit dem State zurückgegeben. Der eigentliche Widget-Tree
    wird dann erst in der \glqq build-Methode\grqq{} der State-Klasse geschrieben und dann auch 
    zurückgegeben. Nach der Erstellung des Widget-Trees kann bei Bedarf durch die Methode 
    \glqq setState\grqq{} das UI aktualisiert werden. Außerdem gibt es asynchrone Methoden und 
    Futures\cite{Futures}. Diese sind dazu da das Programm weiterhin Code ausführen zu lassen,
    während das Programm gleichzeitig darauf wartet, dass die asynchrone Methode fertig wird. 
    Dies wird oft dazu genutzt, Daten vom Netzwerk zu laden oder in eine Datenbank zu 
    schreiben. Die meistgenutzten Datentypen sind: \texttt{int}, \texttt{String}, \texttt{bool}, \texttt{List}, 
    \texttt{Future} und \texttt{Map}.
    Nach einiger Überlegung und Recherchen ist die Entscheidung letztendlich für Flutter als 
    Framework gefallen, weil zum einen nur einmal Code für die beiden Plattformen iOS und 
    Android geschrieben werden muss und zum anderen das GUI einfacher programmierbar ist als 
    z.B. mit Xamarin Forms und Flutter sehr performant ist. In dieser Version sind die Funktionen
    der Alarmierung, des Alarm Empfangens, der Vertretung für den aktuellen Tag, das Wählen von 
    Notfallnummern, die News und die Einstellungen implementiert.