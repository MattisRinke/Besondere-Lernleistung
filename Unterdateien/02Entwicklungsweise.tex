\section{Wahl der Entwicklungsweise}
Es gibt in der Programmierung etliche Möglichkeiten der Programmierung. Auch in der Entwicklung für mobile Endgeräte.
Als ich mit dem Projekt angefangen habe, habe ich zunächst den mir am sinnvollsten erscheinenden Weg genommen. Die native Entwicklung.
Dadurch, dass die App für unterschiedliche Betriebssysteme erhältlich sein soll, muss dies in dem Fall dann zweimal geschehen.
Einmal für das Betriebssystem Android\cite{Android}, von Google, und für das Betriebssystem iOS\cite{iOS} von Apple.
Es gibt aber auch die Möglichkeit der Cross-Platform-Programmierung, bei der für beide Betriebssysteme gleichzeitig programmiert wird.
Ich habe mich zunächst für die native Programmierung entschieden, jedoch habe ich mich im Entwicklungsprozess von der nativen Entwicklung zur
Cross-Platform-Programmierung mit dem Framework Xamarin-Forms bewegt um schließlich das Framework Flutter zu verwenden.
Was die Vor- und Nachteile sind und warum ich mich letztendlich für die Cross-Platform-Programmierung mit Flutter entschieden habe
erkläre ich in den nächsten drei Abschnitten.

    \subsection{Version 1 - Native Entwicklung}
    %Beginn mit der nativen Entwicklung beschreiben
    %Erklären, warum es keinen Sinn macht als Einzelner Programmierer native Entwicklung zu nutzen
    \subsection{Version 2 - Xamarin Forms}
    %Cross-Platform-Programmierung anreißen
    %Erklären, warum Xamarin-Forms nichts für mich war --> Umständliche UI-Programmierung
    \subsection{Version 3 - Flutter}
    %Flutter vorstellen
    %Vorteile von Flutter erklären