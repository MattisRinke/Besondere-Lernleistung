\section{Wahl der Entwicklungsweise}
Es gibt in der Programmierung etliche Möglichkeiten der Programmierung. Auch in der Entwicklung für mobile Endgeräte.
Als ich mit dem Projekt angefangen habe, habe ich zunächst den mir am sinnvollsten erscheinenden Weg genommen. 
Die native Entwicklung. Dadurch, dass die App für unterschiedliche Betriebssysteme erhältlich sein soll, 
muss dies in dem Fall dann zweimal geschehen. Einmal für das Betriebssystem Android\cite{Android}, von Google, und für
das Betriebssystem iOS\cite{iOS} von Apple. Es gibt aber auch die Möglichkeit der Cross-Platform-Programmierung, bei 
der für beide Betriebssysteme gleichzeitig programmiert wird. Ich habe mich zunächst für die native Programmierung 
entschieden, jedoch habe ich mich im Entwicklungsprozess von der nativen Entwicklung zur Cross-Platform-Programmierung 
mit dem Framework Xamarin-Forms bewegt um schließlich das Framework Flutter zu verwenden. Was die Vor- und Nachteile
sind und warum ich mich letztendlich für die Cross-Platform-Programmierung mit Flutter entschieden habe erkläre ich in 
den nächsten drei Abschnitten.

    \subsection{Version 1 - Native Entwicklung}
    Zunächst habe ich mit der nativen Entwicklung von Android begonnen, da ich die eine garantierte Kompatibilität
    mit dem gewünschten Betriebssystem habe. Dies stellte sich als einfach heraus, da ich hier in der 
    Programmiersprache Java schreiben muss, die ich bereits aus dem Informatik-Unterricht kannte.
    Hier hatte ich dann nach einiger Zeit eine vor**läufig fertige App programmiert, in welcher
    man einen Alarm auslösen und empfangen konnte.
    Da der Markt zwischen Apple und Google in Sachen Handy-Betriebssysteme bei ca. 70\% zu 30\% 
    liegt\cite{Marktanteil}, habe ich schnell gemerkt, dass ich die App auch für iOS entwickeln muss.
    Um alle Funktionen immer auf jeder Platform verfügbar zumachen muss bei dieser Entwicklungsmethode
    jede Funktion zweimal programmiert werden. Dies ist für mich als Einzelperson nicht machbar, weshalb
    ich mich neu orientieren musste. Ich habe mich weiter informiert und habe dann die Methode des 
    Cross-Platform-Programmings gefunden, auf welche ich in den nächsten zwei Abschnitten eingehen werde.
    \subsection{Version 2 - Xamarin Forms}
    Xamarin Forms ist ein Framework der .NET-Platform von Microsoft. Dieses Framework ist ein 
    Cross-Platform-Framework, das heißt, dass der Code einmalig für die beiden Betriebssysteme (Android \& iOS)
    geschrieben wird und die App dann für beide erhältlich ist.
    Xamarin Forms hat eine Unterteilung zwischen dem funktionalen Code, welcher in C\# geschrieben ist, und zwischen
    der der Markup Language XAML, welche das GUI darstellt\cite{Xamarin}.
    Jedoch sind bei der Nutzung von Cross-Platform-Frameworks auch Einbußungen zu machen. In diesem Fall konnte 
    ich mich zum einen nicht mit der Markup Language XAML anfreunden, zum anderen musste ich jedoch auch 
    herausfinden, dass Xamarin einige von mir benötigte Funktionen nicht voll oder gar nicht unterstützt. 
    Unter anderem gab es immer wieder Probleme beim einbinden von Firebase-Messaging, ein Tool von Google,
    zum versenden von Push-Notifications (Auf Firebase-Messaging gehe ich im Laufe der Ausarbeitung noch ein).
    Durch diese für mich nicht lösbaren Probleme musste ich mich dann erneut auf die Suche nach einer anderen
    Lösung machen. Um diese Lösung geht es jetzt im nächsten Abschnitt.
    \subsection{Version 3 - Flutter}
    Die dritte und bis jetzt finale Version ist in Flutter geschrieben und die am weitesten entwickelte App-Version.
    Die Entscheidung für Flutter viel nach mehreren Empfehlungen, sowie nach eigener Recherche. 
    Flutter nutzt eine einfache Programmiersprache, die Java ähnelt, weshalb mir der Einstieg in die
    Programmiersprache Dart\cite{Dart} nicht schwer gefallen ist. Flutter ist wie Xamarin-Forms ein 
    Cross-Platform Framework. Dieses ist in der Lage Apps für die beiden gängigen Plattformen iOS und Android 
    zu kompilieren\footnote{Kompilieren beschreibt das Umwandeln des geschriebenen Programmtextes in ein 
    funktionsfähiges Programm}, sowie für das Web.
    Das Nutzer-Interface wird in Flutter durch so genannte Widgets, welche in einem Widget-Tree 
    aufgenommen werden, dargestellt. Hierbei wird dann zwischen Stateless\cite{Stateless-Widget}-
    und Stateful\cite{Stateful-Widget}-Widgets unterschieden. Der Unterschied hier ist, dass in Stateless-Widgets
    zwar Variablen, etc. abgeändert werden können, jedoch wird das Widget im UI nicht aktualisiert. 
    Es können aber in dem Widget selbst andere Widgets, die selbst Stateful-Widgets sind, aktualisiert werden.
    Ein Stateful-Widget besteht aus zwei Klassen. Der Klasse, die um die Klasse Stateful-Widget erweitert wird 
    und der Klasse, die um die Klasse State erweitert wird. In der Klasse mit dem Stateful Widget wird in der
    createState-Methode die Klasse mit dem State zurückgegeben. Der eigentliche Widget-Tree wird dann erst in
    der build-Methode der State-Klasse geschrieben und dann auch zurückgegeben.
    Nach dem der Widget-Tree dann erstellt wurde kann bei Bedarf durch die Methode setState das UI aktualisiert werden.
    Außerdem gibt es asynchrone Methoden und Futures\cite{Futures}.
    Diese sind dazu da das Programm weiterhin Code ausführen zu lassen, während dieses gleichzeitig darauf 
    wartet, dass die asynchrone Methode fertig wird. Dies wird oft dazu genutzt Daten vom Netzwerk zu laden
    oder in eine Datenbank zu schreiben. Die meist genutzten Datentypen sind: int, String, bool, List, Future und Map.
    Ich habe mich nach einiger Überlegung und Recherche dann letztendlich für Flutter als Framework 
    entschieden, weil ich zum einen nur einen Code für die beiden Plattformen iOS und Android schreiben muss 
    und zum anderen da das UI einfach programmierbar ist und Flutter sehr performant ist.