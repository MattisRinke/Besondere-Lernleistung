\section{Schultests}
Um die Funktionalität der App sicherzustellen wird die App zur Zeit an mehreren Schulen getestet.
Dazu wird für die Schulen ein Sanitätsdienst erstellt, über welchen diese dann alle Funktionen testen können.
Sollte dann ein Fehler auftreten können die Nutzer/-innen diesen über ein Fehler-Formular melden. 
Jedoch kann man nicht nur Fehler melden sondern auch generell Feedback, über ein dediziertes Fehler-Formular, abgeben.
Dies dient der allgemeinen Verbesserung der App in jeglichen Belangen, seien es fehlende Funktionen, Design-Abänderungen oder nur Lob.
Derzeit testen bereits 5 Schulen die App, wobei generell positives Feedback zurückkam. Jedoch trat natürlich auch schon der ein oder andere Fehler auf, 
der dann jedoch durch schnelle Kommunikation mit den jeweiligen Ansprechpartner/-innen der Schulen behoben werden konnte.
Zum Beispiel wurde bereits die Wahrscheinlichkeit erhöht, in der ein Alarm ankommt. Hier gab es vor allem am Anfang größere Probleme, da immer wieder Sonderfälle aufgetreten sind, 
durch die Fehler aufgetreten sind. Dadurch, dass in der App dann Fehler aufgetreten sind konnte der Alarm nicht abgespielt und angezeigt werden.
Dies konnte dann durch die Rückmeldungen der Schulen behoben werden.
Andere Fehler waren beispielsweise auch Grafikfehler, bei denen dann gewisse Grafikobjekte nicht korrekt angezeigt wurden oder abgeschnitten waren. Dies konnte dann auch 
schnell durch die Rückmeldungen behoben werden.