\section{Schultests}
Um die Funktionalität der App sicherzustellen, wird die App zur Zeit an 
mehreren Schulen getestet. Dazu wird für die Schulen in der App ein Sanitätsdienst erstellt,
über welchen diese dann alle Funktionen testen können. Sollte dann ein Fehler
auftreten, können die Nutzer/-innen diesen über ein Fehler-Formular von Google Forms 
melden, welches dann per E-Mail an mich versandt wird. 
Jedoch können nicht nur Fehler gemeldet, sondern es kann auch generell Feedback, über ein
dediziertes Formular, abgeben werden. Dies dient der allgemeinen Verbesserung
der App in jeglichen Belangen, seien es fehlende Funktionen, Design-Änderungen
oder nur Lob. Derzeit testen 6 Schulen die App, wobei generell positives
Feedback zurückkam. Natürlich trat auch schon der ein oder andere Fehler 
auf, der dann durch schnelle Kommunikation mit den jeweiligen 
Ansprechpersonen der Schulen behoben werden konnte.
Zum Beispiel wurde bereits die Zuverlässigkeit erhöht, mit der ein Alarm 
ankommt. Hier gab es vor allem am Anfang größere Probleme, da immer wieder 
Sonderfälle (z.B. wurde auf manchen Geräten die Berechtigung für die Batterie-Optimierung
mehrmals abgefragt, obwohl die Berechtigung schon gegeben wurde) aufgetreten sind, durch 
die Fehler aufgetreten sind. Dadurch, dass in der App dann Fehler (z.B. gab es auch Probleme
mit dem versenden von http-Anfragen, die dem Server zurückmelden, dass der Alarm empfangen wurde) 
aufgetreten sind, konnte der Alarm nicht abgespielt und angezeigt werden. Dies konnte dann 
durch die Rückmeldungen der Schulen behoben werden. Andere Fehler waren beispielsweise 
auch Grafikfehler, bei denen dann gewisse Grafikobjekte nicht korrekt angezeigt wurden 
oder abgeschnitten waren. Diese Grafikfehler sind aufgrund unterschiedlicher Handygrößen aufgetreten,
die durch meine Tests nicht aufgefallen sind. Dies konnte ebenfalls durch persönliche Rücksprache behoben werden.
Des weiteren wurde auch ein Fehler beim Anmelden gemeldet, durch welchen
man sich weiterhin registrieren konnte, jedoch eine Anmeldung mit einem 
existenten Account nicht möglich war.
Durch mehrere schnelle Tests war dann schnell klar, dass der Fehler durch die 
falsche Verarbeitung der Serverantwort aufgetreten ist, sodass auch dieser 
schnell behoben werden konnte. 
Diese effektive Fehlerbehebung war zum einen durch das bereits erwähnte 
Fehlerformular möglich, jedoch auch durch einen weiteren Service von Firebase.
Durch den Fehler-Analyse-Dienst \glqq Firebase-Crashlytics\grqq{} ist es möglich, dass Fehler, die in der App 
auftreten auf einer Web-Oberfläche angezeigt werden. Durch die Implementation 
von Firebase-Crashlytics ist durch die, von dem Dienst generierten, Fehler-Meldungen eine Lösungsfindung 
deutlich einfacher, da hier der Fehler zum einen aufgezeigt wird und zum anderen
angezeigt wird, an welcher Stelle dieser ist. Dadurch ist es möglich Fehler, die zwar auftreten, jedoch durch den/die
Nutzer/-in nicht gemeldet werden, weil diese/-r den Fehler ggf. nicht bemerkt hat, behoben werden.