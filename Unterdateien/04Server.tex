\subsection{Kommunikation mit dem Server}
Um alle Funktionen anbieten zu können müssen einige Funktionen auf einen Server
ausgelagert werden. Unter anderem werden hier die Sanitätsdienste mit ihren 
Sanitäter/-innen und Alarmierenden verwaltet. Im Folgenden werde ich nun zuerst
die Nutzung der API (Application Programming Interface; Dieses stellt eine Schnittstelle zwischen 
mehreren Anwendungen zur Verfügung) darlegen und im Anschluss Firebase-Messaging erklären, sowie auf
die Implementierung von Firebase-Messaging eingehen.
\subsubsection{API-Nutzung}
Die API funktioniert so, dass zunächst eine Request, also eine https-Anfrage,
an den Server gesendet wird. Hierbei wird zunächst der Pfad (/path) und 
danach die Attribute in der query \\(?attribut1$=$wert1\&attribut2$=$wert2) 
angegeben. Diese bestehen meistens aus der Sanitätsdienst-ID, der Nutzer-ID, 
der Nutzer-Rolle und weiteren spezifischen Attributen, je nach Daten, die 
abgefragt werden sollen.

Der Aufbau einer solchen Anfrage würde dann wie folgt aussehen: 
%Bild einfügen: https://www.dmergency.de/Alarmfeedback?schoolid=2&ssdid=2&typeid=0&alid=4
%               schema      domain          path        attributes      
Nach dem Empfang der Daten vom Server wird die Antwort des Servers zunächst 
in einem JSON-Array (JSON ist ein Datenformat, in welchem Daten gespeichert werden können.
Ein JSON-Array setzt sich entweder aus Position und Wert, oder aus Schlüsselwort und Wert zusammen.
Zum Beispiel: 0: Wert1) gespeichert um dann weiter verarbeitet zu werden. Dieser ist
dann je nach Art der benötigten Daten 1-3 Dimensional. Zum Beispiel ist der 
JSON-Array für die Berechtigungen 1 Dimensional, der JSON-Array für die 
Sanitäter/-innen, die aktuell Dienst haben 2 Dimensional usw.

Durch die API ist die Programmierung deutlich einfacher und die App kleiner, da
ich so alle Daten, die für die Nutzer/-innen nicht direkt relevant sind auf dem 
Server abspeichern kann, beziehungsweise diese auch abfragen oder ändern kann,
wodurch auf dem Handy selbst weniger Speicherplatz benötigt wird, da hier weniger 
Daten lokal gespeichert werden und weniger Systemressourcen benötigt werden, da
mehrere Prozesse auch durch den Server erledigt werden.



\subsubsection{Nutzung von Firebase-Messaging}
Firebase-Messaging wird genutzt, um die Alarme, die ausgelöst und vom Server
verarbeitet werden, vom Server an die Smartphones der Sanitäter/-innen zu 
schicken. Hierzu habe ich mich entschieden, da Firebase-Messaging mit den 
beiden Plattformen, iOS und Android, für welche auch meine App erhältlich ist, 
kompatibel ist und daher ohne Probleme Benachrichtigungen an die Smartphones
der Sanitäter/-innen geschickt werden können. Da der Server dies auch 
unterstützt und dafür die Möglichkeit bereitstellt, war dies schnell umgesetzt.

Firebase-Messaging ist ein Cloud-Messaging Service von Google. Dieser 
Service ist in der Lage, Benachrichtigungen an Clients (Hier sind die jeweils installierten Apps
der Nutzer/-innen gemeint) zu schicken. Dieses erfolgt
über einen sogenannten FCM(Firebase-Cloud-Messaging)-Token, welcher einem 
spezifischen Gerät bei der Installation der App zugewiesen wird. Jeder FCM-Token
ist einzigartig und wird von Zeit zu Zeit auf jedem Gerät aktualisiert.
\cite{FCM-Update} Das Aktualisieren des Tokens kann durch mehrere Ereignisse 
ausgelöst werden. Zum Einen kann dies dadurch ausgelöst werden, wenn die App auf 
einem anderen Gerät wieder hergestellt wird, oder der/die Nutzer/-in die App 
deinstalliert bzw. diese reinstalliert, oder wenn der/die Nutzer/-in die App-Daten löscht.

Firebase-Nachrichten sind grundlegend immer gleich aufgebaut. 
Grundlegend sind Firebase Nachrichten JSON-Arrays. Diese haben immer einen
message-Teil, welcher dann noch weiter aufgeteilt werden werden. Ein wichtiger 
Teil, der in fast allen Nachrichten enthalten ist, ist der notification-Teil. 
In diesem wird dann der Notification-title und Notification-body angegeben, 
welche in der Notification angezeigt wird. Außerdem gibt es den data-Teil, 
welcher für die Datenübermittlung zwischen Gerät und Server wichtig ist, da 
dieser selbst gestaltet werden kann. Durch den Server ist hier vorgegeben, 
dass bei jedem Alarm eine AlarmId, ein Alarm-Ort, eine Alarm-Beschreibung,
eine Alarm-Priorität, die Zeit der Alarmauslösung, sowie das Datum von dem 
Alarm in dem Data-Teil des JSON-Arrays mitgegeben werden.

Firebase-Messaging stellt jedoch auch, wie sich im Laufe der Arbeit auch noch
herausstellen wird, eine Herausforderung dar, da der Programm-Code, welcher die
Notification im Hintergrundmodus der App (also,  wenn die App gerade nicht 
geöffnet oder nur im Hintergrund geöffnet ist) empfängt und verarbeitet, in einer 
anderen Isolate (Eine Isolate lässt sich mit Threads anderer 
Programmiersprachen und Betriebssysteme vergleichen\cite{Isolates}) ist, als der Code, 
der den Vordergrund-Teil der App verarbeitet, wodurch unter Anderem nicht 
auf die gleichen Daten zugegriffen werden kann.

Um Firebase-Messaging zu nutzen muss zunächst auf der Firebase-Webseite ein Projekt 
konfiguriert werden, in welchem festgelegt wird, welche Apps dem Projekt angehören 
und die Nachrichten vom Server erhalten. Danach müssen die Apps, die in diesem 
Projekt genutzt werden sollen, konfiguriert werden. Um eine Android-App zu 
Konfigurieren, muss der Android-Paketname und ein Name zur Identifizierung 
der App angegeben werden. Im Anschluss muss dann eine Konfigurationsdatei 
(google-services.json) heruntergeladen werden und dem Flutter-Projekt hinzugefügt 
werden (siehe Abbildung).

Das gleiche muss auch für iOS gemacht werden, nur, dass der Ort und die Art der 
Konfigurationsdatei variiert (siehe Abbildung). Bei iOS ist es somit eine 
Info.plist-Datei anstatt der google-services.json-Datei.

Danach muss noch jeweils plattformspezifischer Code eingefügt werden, worauf 
ich jetzt jedoch nicht weiter eingehen werde, da dieser von Firebase selbst 
vorgegeben wird und nur an der richtigen Stelle eingefügt werden muss.
Um jetzt jedoch Firebase-Messaging nutzen zu können muss zunächst im Code die 
Firebase-App initialisiert werden und danach ein Background-Prozess gestartet 
werden, damit die Alarme, auch wenn die App geschlossen ist empfangen werden können.

Um diesen Hintergrundprozess zu starten muss eine Methode erstellt werden, die die 
empfangenen Nachrichten verarbeitet. Dafür muss in der Methode 
FirebaseMessaging.onBackgroundMessage() eine globale Methode übergeben werden. 
Außerdem erwartet das Programm dauerhaft eine Änderung in der Methode FirebaseMessaging.onMessage.
Dies geschieht durch den Aufruf von \glqq .listen\grqq{} auf die Methode 
FirebaseMessaging.onMessage. In der Methode listen() muss dann erneut die Nachricht, 
welche vom Server gesendet wurde, verarbeitet werden. Außerdem muss auf die Methode 
Firebase.instance.onTokenRefresh gehört werden, um die Verarbeitung zu regeln, wenn
der App ein neuer Token zugewiesen wird. Als letzte Methode muss dann noch auf die 
Methode FirebaseMessaging.onMessageOpenedApp() gehört werden, sodass verarbeitet 
werden kann, wenn die App über das Klicken auf die Benachrichtigung geöffnet wird.

In diesem Projekt wird, wenn der Token sich verändert, der Token an den Server übermittelt um
den gespeicherten Token zu aktualisieren. Dies muss durchgeführt werden, damit der/die Nutzer/-in
jederzeit über den neuen Token Alarme empfangen kann. Dies würde nicht möglich sein, sollte der 
gespeicherte Token nicht mehr zu dem von der App zugewiesenen Token passen.

In der listen-Methode von FirebaseMessaging.onMessageOpenedApp() wird zurzeit als 
einziges der Alarmsound gestoppt. Ich habe mich dazu entschieden dies hier einzubauen, 
da die Sanitäter/-innen meist ihr Handy zunächst entsperren und im Anschluss dann auf 
die Benachrichtigung gucken. Es ist außerdem möglich den Alarmsound durch das Betätigen der
Alarmrückmeldungs-Buttons zu stoppen.

In der Methode onBackgroundMessage müssen deutlich mehr Daten verarbeitet werden. 
Zuallererst wird überprüft, ob die Nachricht das Attribut \glqq Backpack\grqq{} mit 
dem Wert \glqq true\grqq{} enthält um zu überprüfen ob dies eine Benachrichtigung darüber 
ist, dass jemand anderes das Material holt, oder ob die Firebase-Nachricht ein Alarm ist. 
Danach müssen die Alarmdaten, welche durch die Firebase-Nachricht empfangen wurden, in die 
Datenbank geschrieben werden, um diese auch später noch anzeigen zu lassen.
Zeitgleich wird außerdem der Alarmsound gestartet, um den/die Sanitäter/-in über den 
Alarm zu informieren.

Zuletzt wird dann eine Request an den Server gesendet, in welcher darüber informiert 
wird, dass der Alarm von dem/der Nutzer/-in empfangen wurde. Dazu wird dem Server die NutzerID,
die SanitätsdienstID und die Rückmeldung als \texttt{int} geschickt, in diesem Fall wird 0 
mitgeschickt, da dies die Information ist, dass der Alarm empfangen wurde.