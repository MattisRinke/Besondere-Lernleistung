\subsection{Kommunikation mit dem Server}
Um alle Funktionen anbieten zu können müssen einige Funktionen auf einen Server ausgelagert werden.
Unter anderem werden hier die Sanitätsdienste mit ihren Sanitäter/-innen und Alarmierenden verwaltet.
Im Folgenden werde ich nun zu erst die API-Nutzung darlegen und im Anschluss Firebase-Messaging erklären, sowie auf die 
Implementierung von Firebase-Messaging eingehen.
\subsubsection{API-Nutzung}
Die API funktioniert so, dass ich zunächst eine Request, also eine https-Anfrage, an den Server sende.
Hierbei spezifiziere ich zunächst den Pfad (/path) und danach werden Attribute in der query \\(?attribut1\=wert1\&attribut2\=wert2) angegeben. Diese bestehen meistens
aus der Sanitätsdienst-ID, der Nutzer-ID, der Nutzer-Rolle und weiteren spezifischen Attributen je nach Daten, die abgefragt werden sollen.

\noindent Der Aufbau einer solchen Anfrage würde dann wie folgt aussehen: 
%Bild einfügen: https://www.dmergency.de/Alarmfeedback?schoolid=2&ssdid=2&typeid=0&alid=4
%               schema      domain          path        attributes      
Nach dem Empfang der Daten vom Server wird die Antwort des Servers zunächst in einem JSON-Array gespeichert um dann weiter verarbeitet zu werden.
Dieser ist dann je nach Art der benötigten Daten 1-3 Dimensional. Zum Beispiel ist der JSON-Array für die Berechtigungen 1 Dimensional, der JSON-Array für die 
Sanitäter/-innen, die aktuell Dienst haben 2 Dimensional usw.



\subsubsection{Nutzung von Firebase-Messaging}
Firebase-Messaging wird genutzt um die Alarme die, ausgelöst und vom Server verarbeitet werden, vom Server an die Smartphones der Sanitäter/-innen zu schicken.
Hierzu habe ich mich entschieden, da Firebase-Messaging mit den beiden Plattformen, iOS und Android, für welche auch meine App erhältlich ist, kompatibel ist und ich daher
ohne Probleme Push-Notifications an die Smartphones der Sanitäter/-innen geschickt werden können. Da der Server, dies auch unterstützt und dafür die Möglichkeit bereitstellt, war dies 
schnell umgesetzt.

\noindent Firebase-Messaging ist ein Cloud-Messaging Service von Google. Dieser Service ist in der Lage Push-Notifications an 
Clients zu schicken. Dieses erfolgt über einen sogenannten FCM(Firebase-Cloud-Messaging)-Token, welcher einem spezifischen Gerät
bei der Installation der App zugewiesen wird. Jeder FCM-Token ist einzigartig und wird von Zeit zu Zeit auf jedem Gerät aktualisiert.\cite{FCM-Update}
Das aktualisieren des Tokens kann durch mehrere Ereignisse ausgelöst werden. Zum einen dies dadurch ausgelöst werden, wenn die App auf einem anderen 
Gerät wieder hergestellt wird, oder der Nutzer die App deinstalliert bzw. diese reinstalliert, oder wenn der Nutzer die App-Daten löscht.

\noindent Firebase-Nachrichten sind grundlegend immer gleich aufgebaut. Grundlegend sind Firebase Nachrichten JSON-Arrays. Diese haben immer einen
message-Teil, dieser kann dann noch weiter aufgedröselt werden. Ein wichtiger Teil, der in fast allen Nachrichten enthalten ist, ist der notification-Teil. 
In diesem wird dann der Notification-title und Notification-body angegeben, welche in der Notification angezeigt wird.
Außerdem gibt es den data-Teil, welcher für die Daten Übermittlung zwischen Gerät und Server wichtig ist, da dieser selbst gestaltet werden kann.
Durch den Server ist hier vorgegeben, dass bei jedem Alarm eine AlarmId, ein Alarm-Ort, eine Alarm-Beschreibung, eine Alarm-Priorität, die Zeit der Alarmauslösung, sowie das Datum von dem Alarm.

