\subsection{Funktionen der App}
    \begin{changemargin}{0,5cm}{0,0cm}
        In den folgenden Abschnitten werden jetzt die Funktionen der App dargestellt und erklärt.
        Dazu werden beispielhaft einzelne Methodenimplementationen herrausgenommen, erörtert und
        im Kontext der jeweiligen Funktion erklärt.   
    \end{changemargin}
    \subsubsection{Rollen und Registrierung}
        \begin{changemargin}{0,5cm}{0,0cm}
            Rolle 1: Alarmierende\;

            Die Rolle Alarmierende/r soll nur dazu in der Lage sein einen Alarm auszulösen, 
            sowie die News für Alarmierende und Notfallnummern einzusehen.\;
            Ein Alarmierender muss bei der Registrierung einen Namen, eine E-Mail-Adresse, sowie ein
            Passwort angeben.
            Die E-Mail-Adresse wird verwendet um den Account innerhalb der App zu identifizieren und 
            den Nutzer in der App einzuloggen. Das Passwort um sich jederzeit an einem Handy
            einloggen zu können. Der Name dient der / den administrierenden Person(en) zur
            Identifikation des Alarmierenden.
        \end{changemargin}

        \begin{changemargin}{0,5cm}{0,0cm}
            Rolle 2: Sanitäter/-in\;

            Die Rolle Sanitäter/-in soll in der Lage sein einen Alarm zu empfangen, einen Alarm
            auszulösen und andere Sanitäter/-innen zu vertreten oder sich im Notfall aus dem Dienst 
            auszutragen.
            Ein/e Sanitäter/-in muss bei der Registrierung seinen/ihren Vornamen,  Nachnamen,
            das Geschlecht, eine E-Mail-Adresse, ein Passwort, sowie eine Stufe angeben.

        \end{changemargin}

        \begin{changemargin}{0,5cm}{0,0cm}
            Um die unterschiedlichen Rollen umzusetzen muss auch der Registrierungsprozess von
            den verschiedenen Rollen verscieden ablaufen.
        \end{changemargin}

    \subsubsection{Alarmauslösung}
    \subsubsection{Alarmempfang}
    \subsubsection{Vertretungen}
    \subsubsection{News}
    \subsubsection{Notfallnummern}