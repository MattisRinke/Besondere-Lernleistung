\section{Fazit}
In den folgenden zwei Abschnitten erfolgt mein Fazit, in dem ich zunächst erläutere, was erreicht wurde und 
im Anschluss einen Ausblick gebe, was noch umgesetzt werden muss, beziehungsweise, welche Funktionen vielleicht noch hinzugefügt werden sollen.
\subsection{Was wurde erreicht}
    Nach 5705 Zeilen Code und ca. 500 Stunden Zeit, die ich in die App gesteckt habe, steht jetzt eine vorerst fertige App.
    Es treten zwar immer noch, wie ich bereits im Kapitel Schultest beschrieben habe, Fehler auf, diese werden
    jedoch nach und nach behoben. Die eben benannten Fehler sind nicht durch eigene Testreihen aufgetreten, weshalb
    ich davon ausgehe, dass dies vereinzelte sind und ich diese sukzessive abarbeiten kann. Da die App jetzt 
    soweit ist, dass sie bald verkauft werden könnte, und dies mein bereits in der Einleitung erwähntes Ziel ist,
    habe ich mich erkundigt, welche Anforderungen es dafür gibt. Da ich mit meiner Tätigkeit darauf abziele regelmäßig Geld
    zu verdienen, fand ich heraus, dass es notwendig ist ein Gewerbe anzumelden.
    Wie bereits am Anfang erwähnt, habe ich mich mit einem Partner zusammengetan, um die Probleme, die ich unter anderem 
    in der Einleitung darstellte, zu lösen. Dieser Partner erstellte den Server für die App, sowie eine Website.
    Somit können wir auch die App, bzw. das gesamt Konstrukt mit App, Website und Server, gemeinsam vermarkten und gründeten
    dafür eine Gesellschaft bürgerlichen Rechts (GbR) und reichten eine Gewerbeanmeldung beim zuständigen Gewerbeamt ein.
    Des weiteren wurde die App mit allen bisher beschriebenen Funktionen in die App Store von Google und Apple geladen, damit die App von Sanitätsdiensten 
    heruntergeladen werden kann. Dies geschieht derzeit bereits an sechs Schulen (Das Gymnasium Nepomucenum; die Nachbarschule, 
    Theodor-Heuss-Realschule Coesfeld, das Gymnasium Johanneum in Ostbevern, die Geschwister-Scholl-Realschule in Senden, 
    die Gesamtschule Bocholt und das St. Antonius Gymnasium in Lüdinghausen), welche mit uns in stetigem Austausch über Probleme und 
    Verbesserungsvorschläge stehen. Außerdem läuft die von meinem Partner entwickelte Website und der Server und somit die Kommunikation bereits über eine eigene Domain,
    um die Seriosität zu steigern. Über diese läuft auch unsere Support-Email-Adresse, über die mit den Sanitätsdiensten der teilnehmenden Schulen kommuniziert
    wird. Insgesamt wurde also im Rahmen dieses Projektes eine GbR gegründet, eine App für iOS und Android programmiert, die durch ein 
    Login- und Registriersystem zwischen Alarmierenden und Sanitäter/-innen unterscheidet, sowie
    die Sanitäter/-innen nach einem Dienstplan alarmieren kann und ein Vertretungssystem hat, damit sich die Sanitäter/-innen 
    gegenseitig vertreten können. Final wurden diese beiden Apps dann auch noch in die beiden Appstores von Google und Apple hochgeladen.
\subsection{Wie geht es weiter}
Nachdem mir nun bereits ein erstes Feedback vorliegt, ist es mein erstes Ziel dieses Feedback einzuarbeiten und zudem einige weitere Funktionen
hinzuzufügen. Zum einen sollen weitere Funktionen hinsichtlich des Dienstplans implementiert werden, damit man sich nicht jedes Mal auf der
Website anmelden muss, wenn man den Dienst im größeren Umfang abändern möchte. Genauer gesagt sollen Funktionen hinzugefügt werden, durch die 
der Dienstplan erstellt und abgeändert werden kann, sowie eine Funktion, durch die man schon im Voraus Vertretungen eintragen kann, wenn man 
bereits früh weiß, dass man für seinen Dienst verhindert ist. Außerdem sollen für die Administrator/-innen mehrere Funktionen eingebaut werden, 
wie zum Beispiel das Verwalten der Sanitäter/-innen und der Alarmierenden in der App, sodass auch in der App Berechtigungen zur Nutzung der App und Daten der 
Nutzer/-innen (wie z.B. den Qualifikationen) angepasst werden können. Darüber hinaus soll das Erstellen und Bearbeiten von News auch in der App möglich sein, um auch hier
eine effektivere Nutzung dieser Funktion zu ermöglichen. Um die Alarmierung noch weiter abzusichern und zuverlässiger zu machen, wird außerdem eine
Funktion implementiert werden, die die FCM-Token des Gerätes und des Server abgleicht und diese dann bei Bedarf abändert.
Angedacht sind des weiteren ein Terminsystem, welches die Termine der Sanitätsdienste verwalten kann. Hier sollen dann Termine angeboten werden 
können und dann durch die Nutzer/-innen bestätigt oder abgelehnt werden können, um zum Beispiel Fortbildungen zu planen.
Die letzte bis jetzt weiter geplante Funktion ist die Einbindung des Feueralarms, welche alle Nutzer/-innen im Falle eines Feueralarms
über diesen Weg informieren soll. Jedoch müssen hier noch einige Überlegungen gemacht werden, da geplant werden muss, das Recht hat den Feueralarm
auszulösen und wie das Problem gelöst werden kann, dass der Alarm im Falle eines Feuers nicht ankommt, weil die Nutzer auf Grund 
des ausgefallenen WLAN Netzes ggf. nicht erreichbar sind. Hier könnte sich ggf. eine Lösung über Bluetooth
anbieten, welche aber in einem aufwendigen Verfahren getestet werden muss.

Wie schon mehrfach erwähnt, soll die App an Sanitätsdienste verkauft werden, weshalb auch das eines der nächsten 
Ziele mit der App ist. Weitere Sanitätsdienste sollen die App nutzen, zunächst wie mit den ersten 6 Schulsanitätsdiensten, kostenlos,
um die Tests auszuweiten, und dann, wenn die App zuverlässig genug läuft, kostenpflichtig, damit sich die mehr als 500 Stunden Arbeit, sowie 
das Geld, welches ich in die App investiert habe (unter anderem habe ich ein MacBook gekauft um auf diesem für iOS programmieren zu können), rentieren. Damit mehr Sanitätsdienste auf die App aufmerksam werden, ist zusätzlich eine Werbewebsite
geplant, auf der die App mit ihren Funktionen vorgestellt werden soll, damit sich die Sanitätsdienste gut über die App informieren können.

Aufgrund des bereits erhaltenen Feedbacks bin ich auch auf das Problem gestoßen, dass Huawei-Geräte mittlerweile von Google-Diensten ausgeschlossen sind.
So auch vom Google-Playstore, weshalb jetzt geplant ist die programmierte App auf den Appstore von Huawei, der Huawei-AppGallery, hochzuladen und dort
für Huawei-Nutzer/-innen erhältlich zu machen.

Ein neues Feld ergab sich im Bereich der Betriebswirtschaft, wo die Wirtschaftsförderung Coesfeld(WfC),
das Finanzamt und das Amtsgericht mit ins Spiel kamen. Hier stellt sich die Frage, ob mein Partner und ich Experten 
mit einbinden, die uns zum Beispiel beim Marketing weiterhelfen, da wir und weiterhin auf das Programmieren der App und der
Website fokussieren möchten. Durch dieses Projekt sieht man, wie eng der Bereich der Betriebswirtschaft mit dem Programmieren und Verkaufen einer
App zusammenhängt. Dies hatten wir zu Beginn nicht so erwartet, um eine App erfolgreich produzieren zu können.

Aus meiner Sicht wurde schon viel erreicht, jedoch sind auch noch lange nicht alle Vorhaben 
umgesetzt, womit bestimmt erneut hunderte Stunden Arbeit gefüllt werden können. Ich habe in dem Prozess der Entwicklung der App viel gelernt, sei es 
über das Cross-Plattform-Framework Flutter oder den Umgang mit Diensten für sogenanntes Version-Control, auf was ich leider in dieser Arbeit nicht weiter eingehen konnte.
Herauszufinden, wie man eine App programmiert, testet, auf die Appstores hochlädt und schließlich wie man eine GbR gründet, um die App zu verkaufen,
hat mir sehr viel Spaß gemacht und viel neues Wissen bereitet, welches ich hoffentlich weiterhin in der Zukunft anwenden kann. Sei es für die 
App selbst oder in einem späteren Beruf.