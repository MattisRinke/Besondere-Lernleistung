\subsection{Kommunikation mit dem Server}
Um alle Funktionen anbieten zu können müssen einige Funktionen auf einen Server
ausgelagert werden. Unter anderem werden hier die Sanitätsdienste mit ihren 
Sanitäter/-innen und Alarmierenden verwaltet. Im Folgenden werde ich nun zu erst
die API-Nutzung darlegen und im Anschluss Firebase-Messaging erklären, sowie auf
die Implementierung von Firebase-Messaging eingehen.
\subsubsection{API-Nutzung}
Die API funktioniert so, dass ich zunächst eine Request, also eine https-Anfrage,
an den Server sende. Hierbei spezifiziere ich zunächst den Pfad (/path) und 
danach werden Attribute in der query \\(?attribut1\=wert1\&attribut2\=wert2) 
angegeben. Diese bestehen meistens aus der Sanitätsdienst-ID, der Nutzer-ID, 
der Nutzer-Rolle und weiteren spezifischen Attributen je nach Daten, die 
abgefragt werden sollen.

\noindent Der Aufbau einer solchen Anfrage würde dann wie folgt aussehen: 
%Bild einfügen: https://www.dmergency.de/Alarmfeedback?schoolid=2&ssdid=2&typeid=0&alid=4
%               schema      domain          path        attributes      
Nach dem Empfang der Daten vom Server wird die Antwort des Servers zunächst 
in einem JSON-Array gespeichert um dann weiter verarbeitet zu werden. Dieser ist
dann je nach Art der benötigten Daten 1-3 Dimensional. Zum Beispiel ist der 
JSON-Array für die Berechtigungen 1 Dimensional, der JSON-Array für die 
Sanitäter/-innen, die aktuell Dienst haben 2 Dimensional usw.

Durch die API ist die Programmierung deutlich einfacher und die App kleiner, da
ich so alle Daten, die für die Nutzer/-innen nicht direkt relevant sind auf dem 
Server abspeichern kann, beziehungsweise, diese auch abfragen oder ändern kann,
wodurch auf dem Handy selbst weniger Speicherplatz benötigt wird, da hier weniger 
Daten lokal gespeichert werden und weniger Systemressourcen benötigt werden, da
mehrere Prozesse auch durch den Server erledigt werden.



\subsubsection{Nutzung von Firebase-Messaging}
Firebase-Messaging wird genutzt um die Alarme die, ausgelöst und vom Server
verarbeitet werden, vom Server an die Smartphones der Sanitäter/-innen zu 
schicken. Hierzu habe ich mich entschieden, da Firebase-Messaging mit den 
beiden Plattformen, iOS und Android, für welche auch meine App erhältlich ist, 
kompatibel ist und ich daher ohne Probleme Push-Notifications an die Smartphones
der Sanitäter/-innen geschickt werden können. Da der Server, dies auch 
unterstützt und dafür die Möglichkeit bereitstellt, war dies schnell umgesetzt.

Firebase-Messaging ist ein Cloud-Messaging Service von Google. Dieser 
Service ist in der Lage Push-Notifications an Clients zu schicken. Dieses erfolgt
über einen sogenannten FCM(Firebase-Cloud-Messaging)-Token, welcher einem 
spezifischen Gerät bei der Installation der App zugewiesen wird. Jeder FCM-Token
ist einzigartig und wird von Zeit zu Zeit auf jedem Gerät aktualisiert.
\cite{FCM-Update} Das Aktualisieren des Tokens kann durch mehrere Ereignisse 
ausgelöst werden. Zum einen dies dadurch ausgelöst werden, wenn die App auf 
einem anderen Gerät wieder hergestellt wird, oder der Nutzer die App 
deinstalliert bzw. diese reinstalliert, oder wenn der Nutzer die App-Daten löscht.

Firebase-Nachrichten sind grundlegend immer gleich aufgebaut. 
Grundlegend sind Firebase Nachrichten JSON-Arrays. Diese haben immer einen
message-Teil, dieser kann dann noch weiter aufgedröselt werden. Ein wichtiger 
Teil, der in fast allen Nachrichten enthalten ist, ist der notification-Teil. 
In diesem wird dann der Notification-title und Notification-body angegeben, 
welche in der Notification angezeigt wird. Außerdem gibt es den data-Teil, 
welcher für die Daten Übermittlung zwischen Gerät und Server wichtig ist, da 
dieser selbst gestaltet werden kann. Durch den Server ist hier vorgegeben, 
dass bei jedem Alarm eine AlarmId, ein Alarm-Ort, eine Alarm-Beschreibung,
eine Alarm-Priorität, die Zeit der Alarmauslösung, sowie das Datum von dem 
Alarm in dem Data-Teil des JSON-Arrays mitgegeben werden.

Firebase-Messaging stellt jedoch auch, wie sich im Laufe der Arbeit auch noch
herausstellen wird eine Herausforderung dar, da der Programm-Code, welcher die
Notification im Hintergrundmodus der App (also,  wenn die App gerade nicht 
geöffnet oder nur im Hintergrund geöffnet ist) empfängt und verarbeitet in einer 
anderen Isolate \footnote{Eine Isolate lässt sich mit Threads anderer 
Programmiersprachen und Betriebssysteme vergleichen} ist, als der Code 
der den Vordergrund-Teil der App verarbeitet, wodurch unter anderem nicht 
auf die gleichen Daten zugegriffen werden kann, etc.

Um Firebase-Messaging zu nutzen muss zunächst auf der Firebase-Seite ein Projekt konfiguriert
werden, in welchem festgelegt wird, welche Apps dem Projekt angehören und die Nachrichten vom 
Server erhalten. 
Danach müssen die Apps, die in diesem Projekt genutzt werden sollen konfiguriert werden. 
Um eine Android-App zu Konfigurieren muss der Android-Packagename und ein Name zur Identifizierung 
der App angegeben werden. Im Anschluss muss dann eine Konfigurationsdatei (google-services.json)
heruntergeladen werden und dem Flutter-Projekt hinzugefügt werden (siehe Abbildung).

Das gleiche muss auch für iOS gemacht werden nur, dass der Ort und die Art der Konfigurationsdatei
variiert (siehe Abbildung).

Danach muss noch jeweils Plattform spezifischer Code eingefügt werden, worauf ich jetzt jedoch nicht weiter 
eingehen werde, da dieser von Firebase selbst vorgegeben wird und nur an der richtigen Stelle eingefügt 
werden muss.
Um jetzt jedoch Firebase-Messaging nutzen zu können muss zunächst im Code die Firebase-App initialisiert werden 
und danach ein Background-Prozess gestartet werden, damit die Alarme auch wenn die App geschlossen ist empfangen 
werden können.

Um diesen Hintergrundprozess zu starten muss eine Methode erstellt werden, die die empfangenen Nachrichten 
verarbeitet. Dafür muss in der Methode FirebaseMessaging.onBackgroundMessage() eine globale Methode mitgegeben 
werden. Außerdem muss auf die Methode FirebaseMessaging.onMessage gelistened werden. Dies geschieht durch den Aufruf 
von \glqq .listen\grqq{} auf die Methode FirebaseMessaging.onMessage. In der Methode listen() muss dann erneut die 
Nachricht, welche vom Server gesendet wurde verarbeitet werden.