\section{Das Projekt}
\subsection{Zielsetzung}
    Die App soll das Alarmieren und Verwalten von Sanitätsdiensten vereinfachen. Um dies
    zu verwirklichen ist es erforderlich mehrere Funktionen zu implementieren. Zum 
    einen ist eine Funktion zum Alarmieren unerlässlich, welche zur Vergewisserung für
    die alarmierende Person auch ein Feedback anzeigen sollte, zum anderen sollte es 
    dann auch eine Funktion zum Empfangen des Alarms geben. Diese beiden Funktionen 
    sollten so implementiert werden, dass eine alarmierende Person so wenig Aufwand wie 
    möglich in der Durchführung der Alarmierung hat, die Sanitäter/-innen jedoch trotzdem 
    so viele Informationen wie möglich bekommen. Damit eine Verständigung seitens der 
    Sanitäter/-innen darüber erzielt werden kann, wer sich für das Einsatzmaterial 
    verantwortlich zeigt, ist es erforderlich, dass dies ebenfalls mit in die Funktionen 
    aufgenommen wird.
    Darüber hinaus ist die Programmierung eines Dienstplan-Systems vorgesehen, das nur
    diensthabende Sanitäter/-innen alarmiert.
    Selbstverständlich gibt es im (Schul-)Alltag auch Situationen, in denen Sanitäter/-innen
    kurzfristig (z.B. im Falle von Krankheiten) oder auch absehbar längerfristig (z.B.
    durch angekündigte Arbeiten, Ausflüge oder Praktika) ihren Dienst nicht durchführen 
    können, so dass es eine Funktion geben muss, mit der sich diese austragen können und
    andere ihren Dienst übernehmen.
    Im Notfall ist eine schnelle Rettungskette von großer Bedeutung. Um diese Hilfskette
    \cite{Rettungskette} einzuhalten, sollen in der App wichtige Notfallnummern hinterlegt
    werden, welche dann durch einen schnellen Klick wählbar sind.
    
    Ein weiterer wichtiger Bestandteil zur Verwaltung des Sanitätsdienstes 
    ist die Kommunikation zwischen der Leitung und den Mitgliedern des Sanitätsdienstes.
    Um diese Kommunikation sicherzustellen soll eine News-Funktion programmiert werden, 
    in welcher die Mitglieder Neuigkeiten von der Leitung einsehen können.
    Hierzu ist es unabdingbar eine Funktion umzusetzen, die es der Leitung ermöglicht,
    \glqq News schreiben\grqq{} zu können.

\subsection{Abgrenzung zur Server Ausarbeitung}
    In dieser Ausarbeitung wird die Funktionsweise der App \glqq DMergency" \grqq{} 
    beschrieben und wie sie in Zusammenarbeit mit dem Server arbeitet. Es wird nicht 
    darauf eingegangen, wie der Server arbeitet und welche Funktionen es in der 
    Web-Anwendung gibt. Außerdem wird die Rolle des/der Administrator/-in mehrfach erwähnt werden.
    Diese ist jedoch zurzeit nur durch die Web-Anwendung nutzbar.
    Zum Teil werden Daten vom Server verarbeitet oder auf diesem 
    gespeichert. In diesen Fällen wird dies erwähnt jedoch nicht weiter auf die
    Verarbeitung eingegangen.