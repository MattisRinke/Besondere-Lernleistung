\section{Das Projekt}
\subsection{Zielsetzung}
    Die App soll das Alarmieren und Verwalten von Sanitätsdiensten vereinfachen. Um dies zu verwirklichen müssen mehrere Funktionen
    implementiert werden. Zum einen muss es eine Funktion zum Alarmieren geben, welche zur Vergewisserung für die Alarmierende Person
    auch ein Feedback anzeigen sollte, zum Anderen sollte es dann logischer Weise auch eine Funktion zum Empfangen des Alarms geben.
    Diese beiden Funktionen sollten so implementiert werden, dass eine alarmierende Person so wenig Aufwand wie möglich beim Alarmieren hat,
    die Sanitäter/-innen jedoch so viele Informationen wie möglich bekommen. Damit die Sanitäter/-innen sich zusätzlich darüber verständigen können, 
    wer das Einsatzmaterial holt, sollte dies auch mit in diese Funktionen aufgenommen werden.
    Um zu definieren, wer einen Alarm wann erhält soll außerdem ein Dienstplan System programmiert werden, damit nicht immer alle Sanitäter/-innen
    alarmiert werden. Da es aber auch immer Notfallsituationen gibt, in denen einzelne Sanitäter/-innen nicht erreichbar sind, erfordert dies, 
    dass eine Funktion eingebaut wird, durch die einzelne Sanitäter/-innen von anderen Sanitäter/-innen vertreten werden können oder sich die Sanitäter/-innen, sollten 
    sie keine Vertretung finden o.ä., austragen können.
    Um die Hilfskette\cite{Rettungskette} möglichst kurz zu halten sollen in der App wichtige Notfallnummern hinterlegt werden, welche dann durch einen
    schnellen Klick auch wählbar sind.
    Ein weiterer wichtiger Bestandteil zur Verwaltung des Sanitätsdiensts ist die Kommunikation zwischen der Leitung und den Mitgliedern des Sanitätsdiensts.
    Um diese Kommunikation sicherzustellen soll eine News-Funktion programmiert werden, in welcher die Mitglieder Neuigkeiten von der Leitung einsehen können.
    Die Leitung muss dann natürlich News schreiben können.

\subsection{Abgrenzung zur Server Ausarbeitung}
    In dieser Ausarbeitung wird die Funktionsweise der App "DMergency" be-schrieben und wie sie in Zusammenarbeit mit dem 
    Server arbeitet. Es wird nicht darauf eingegangen, wie der Server funktioniert und welche Funktionen es in der Web-Anwendung gibt.
    Zum Teil werden Daten vom Server verarbeitet oder auf diesem gespeichert. In diesen Fällen wird dies erwähnt jedoch nicht weiter auf 
    die Verarbeitung eingegangen.