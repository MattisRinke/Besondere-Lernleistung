\subsection{Speicherung der Daten}
\subsubsection{Umsetzung}
\begin{changemargin}{0,5cm}{0,0cm}
    Es gibt mehrere Möglichkeiten auf mobilen Endgeräten appspezifische Daten zu speichern.
    Zum einen gibt es die so genannten SharedPreferences bzw. NSUserDefaults
    \footnote{SharedPreferences (Android), bzw. NSUserDefaults(iOS) ist platformspezifischer Langzeitspeicher für einfache Daten (String, Integer)}
    dies sind einfache Schlüssel, mit denen ein Wert verknüpft wird.
    Eine weitere Möglichkeit ist eine lokale Datei, in welche alle wichtigen Daten geschrieben werden, oder als letzte, dritte Möglichkeit gibt es
    die Datenbank.

    Im Fall der App DMergency habe ich zunächst versucht die Platform-Nativen Speichermethoden, also die SharedPreferences bzw. die NSUserDefaults, zu nutzen.
    Dies habe ich gemacht, da ich in der App selbst eigentlich kaum Daten speichern muss, was im Laufe der Ausarbeitung noch deutlich wird.
    Da die Platform-Nativen Speichermethoden so einfach gehalten sind und in der App nur Strings und Integer gespeichert werden müssen bieten sich diese also sehr gut an.

    Dieser Gedanke musste jedoch schnell verworfen werden, da die Platform-Nativen Speichermethoden unterschiedliche Instanzen in unterschiedlichen Threads haben.
    \\
    \\Dies wirft zwei Probleme auf:
    \item 1. Die Effizienz der App wird stark beeinträchtigt
    \item 2. Daten können in unterschiedlichen Threads nicht mit den Platform-Nativen Speichermethoden abgerufen werden.
    \\\\Um Alarme zu empfangen muss ein Hintergrundprozess laufen, welcher in einem anderen Thread arbeitet als der Rest der App.
    Da jedoch auf die Alarmdaten auch im Rest der App zugegriffen werden muss und auch im Hintergrundprozess auf die gespeicherten Nutzerdaten zurückgegriffen werden muss, sind die
    platformspezifischen Speichermethoden für die App nicht in Frage gekommen.

    Daher habe ich mich dazu entschieden eine Datenbank anzulegen, welche als lokale Datei auf dem Handy abgelegt ist.
    Dadurch ist es mögich jederzeit auf alle Daten zuzugreifen und die Probleme, der Effizienz und der Speicherung der Daten sind behoben.
    

\end{changemargin}
\subsubsection{Aufbau der Datenbank}